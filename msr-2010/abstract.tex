Modern software is often developed over many years with hundreds of
thousands of commits. Commit metadata is a rich source of social
characteristics, including the commit's time of day and the
experience and activity level of its author.  The ``bugginess'' of a
commit is also a critical property of that commit.  In this work, we
investigate the correlation between a commit's social characteristics
and its ``bugginess''; such results can be very useful for software
developers and software engineering researchers. For instance,
developers or code reviewers might be well-advised to thoroughly
verify commits that are more likely to be buggy.

In this paper, we study the correlation between a commit's bugginess
and the time of day of the commit, the day of week of the commit, and
the experience and activity frequency of the commit authors.  We
survey two widely-used open source projects: the Linux kernel and
PostgreSQL.  

Our main findings include: (1) commits submitted between midnight and
4 AM (referred to as late-night commits) are significantly buggier 
and commits between 7AM and noon are less buggy, implying that developers 
may want to double check their own late-night commits; 
(2) daily committing developers produce less-buggy commits, indicating that we may 
want to promote the practice of daily committing developers code-reviewing other 
developers' commits; and (3) 
the day of week of commits
vary for different software projects, implying that the bugginess prediction based on 
the day of week of commits may need to be on a project-by-project basis.

%We have discovered several interesting findings, and we
%present our findings and their implications in this paper.


