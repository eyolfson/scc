Modern software is often developed over many years with hundreds of
thousands of commits. Commit metadata is a rich source of social
characteristics, including the commit's time of day, and the
experience and activity level of its author.  The ``bugginess'' of a
commit is also a critical property of that commit.  In this work, we
investigate the correlation between a commit's social characteristics
and its ``bugginess''; such results can be very useful for software
developers and software engineering researchers. For instance,
developers or code reviewers might be well-advised to thoroughly
verify commits that are more likely to be buggy.

In this paper, we study the correlation between a commit's bugginess
and the time of day of the commit, the day of week of the commit, and
the experience and activity frequency of the commit authors.  We
survey two widely-used open source projects: the Linux kernel and
PostgreSQL.  We have discovered several interesting findings, and we
present our findings and their implications in this paper.

%% Our main findings include: (1) commits submitted between midnight and
%% 3 AM are significantly buggier and commits submitted between 7-9 AM
%% are less buggier, implying that developers may want to double check
%% their own midnight commits; and (2) daily commiting developers produce
%% less buggy commits, indicating that we may want to have daily
%% commiting developers code-review other developers' commits.

\comment{There are many factors which may contribute to patch quality, our goal
is to find these correlations. Previous studies used day of the week
in order to find correlations. We extend this by using time of the day
and author frequency/experience. We discuss the impact of our work
through motivating examples, which may lead to higher quality code.

We implemented a tool for software repositories which determines
bug-fixing and bug-introducing commits. We describe our implementation
and evaluate it on open source projects. Finally, we present our
findings from these projects.}

%% If we need more information...
%%
%% First, given a software repository, we extract bug-fixing commits
%% using a keyword search. Then, we automatically determine the
%% bug-introduction commits from this using ``blame''
%% information. Finally, we store this in a database and analyze the
%% commit and author information.
