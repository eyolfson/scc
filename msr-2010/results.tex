\section{Results}
\label{sec-results}
In this section, we present the results obtained from carrying out our
methodology, and discuss some of the implications of our results.
Most of our results investigate the effect of an independent variable
(time-of-day and developer experience/frequency classifications) on
the likelihood of a commit to be a bug-introducing commit, or
\emph{bugginess}. We also
describe our findings with respect to the day of the week, which
allows us to compare our results to those
in~\cite{sliwerski-msr-2005}.  Finally, we explain the precision and
recall of our methodology and how we computed these figures.

\paragraph{Project characteristics}
We chose two large open-source software repositories for our
investigations: Linus Torvalds's mainline Linux kernel, from
\url{git.kernel.org} and PostgreSQL, from the project's repository at
\url{git.postgresql.org}.  Table~\ref{tab:characteristics} summarizes
the characteristics of each of our repositories. Note that the
PostgreSQL repository was converted from CVS using {\code cvs2git} in
September 2010.

\begin{table}
\begin{tabular}{lrr}
& {\bf Linux} & {\bf PostgreSQL} \\
First commit & April 16, 2005 & July 9, 1996 \\
Lines of code at tip & over 5 million & over 750,000 \\
%Source & \url{git.kernel.org} & \url{git.postgresql.org} \\
Cloned & November 21, 2010 & January 24, 2011 \\
Number of authors & 8,594 & 34 \\
Number of commits & 222,332 & 31,098 \\
\# bug-introducing & 56,681 (25\%) & 7,366 (23.7\%) \\
\# bug-fixing & 57,028 & 4,399
\end{tabular}
\caption{\label{tab:characteristics}Characteristics of Linux kernel and PostgreSQL repositories.}
\end{table}

%% The Linux repository
%% was cloned on November 21, 2010, from Linus Torvalds's mainline
%% kernel, hosted at \url{git.kernel.org}; this repository contains
%% history back to April 16, 2005.  This repository contains 222,332
%% commits contributed by 8,594 authors. Of these commits, we identified
%% 56,681 bug-introducing commits and 57,028 bug-fixing commits.
%% The tip of the repository contains
%% over 5 million lines of code. 

%% The PostgreSQL repository was cloned on January 24, 2011, from the
%% project's repository at \url{git.postgresql.org}; it contains history
%% to July 9, 1996, translated from CVS using \code{cvs2git}.  This
%% repository contains 31,098 commits contributed by 34 authors. We
%% identified 7,366 bug-introducing commits and 4,399 bug-fixing commits. The tip
%% of the repository contains over 750,000 lines of code.

\subsection{Time-of-day Results} 
We first present our results correlating the time-of-day of a commit
with its bugginess.  Figures~\ref{fig-linux-bugginess-hour}
and~\ref{fig-postgresql-bugginess-hour} present results from Linux and
Postgres, respectively. These graphs compare the time-of-day of each
commit, in the committer's local time on a 24-hour clock, to the
percentage of commits which were found to be bug-introducing. The
solid line indicates the mean number of buggy commits in each project;
bars shorter than the line indicate that commits at that hour were
less likely to be buggy, while bars taller than the line indicate
more-buggy commits.

Both figures show a noticeable increase in the amount of commits which
introduce a bug between 00:00 (midnight) and 04:00 (4AM). After 04:00,
commits tend to be less buggy than average, gradually increasing until
noon.  In Linux, commits between noon and midnight fluctuate around
the average bugginess level, while the Postgres commits are generally
above the average bugginess level between 16:00 (4PM) and 20:00 (8PM),
and then below the average bugginess level between 20:00 (8PM) and
00:00 (midnight).

Our results suggest that tired developers (midnight-4AM) are more
likely to miss corner cases in a pre-commit review. Furthermore, we
can observe that commits before noon are least likely to be
bug-introducing; perhaps committers are most careful in those hours.

\begin{figure}
\begin{center}
\includegraphics[width=0.45\textwidth]{linux-bugginess-hour.pdf}
\end{center}
\caption{Linux: time of day versus percentage of buggy commits}
\label{fig-linux-bugginess-hour}
\end{figure}

\begin{figure}
\begin{center}
\includegraphics[width=0.45\textwidth]{postgresql-bugginess-hour.pdf}
\end{center}
\caption{PostgreSQL: time of day versus percentage of buggy commits}
\label{fig-postgresql-bugginess-hour}
\end{figure}

%% \begin{figure}
%% \begin{center}
%% \includegraphics[width=0.45\textwidth]{linux_per_class.png}
%% \end{center}
%% \caption{Linux percentage of bug introductions and percentage of total commits per author classification}
%% \label{fig-linux-class}
%% \end{figure}

%% \begin{figure}
%% \begin{center}
%% \includegraphics[width=0.45\textwidth]{firefox_per_class.png}
%% \end{center}
%% \caption{Firefox percentage of bug introductions and percentage of total commits per author classification}
%% \label{fig-firefox-class}
%% \end{figure}

\subsection{Developer Characteristics}
We next present our findings with respect to developer activity and
experience.  Developer activity investigates the frequency of a developer's
contributions to a project, while developer experience tracks how long
a developer has contributed to a particular project.


\paragraph{Developer Activity Classification} 
We then investigated how each classification of developers fare in
terms of their likelihood of creating a
bug. Figures~\ref{fig-linux-class} and~\ref{fig-firefox-class} present
our results from Linux and Firefox, respectively.  They show that
developers who commit changes daily, but not as their day job,
contribute significantly to both projects. We believe that this
characteristic should be shared by most open-source projects. Also,
note that for both projects, the daily developers are less likely to
produce bugs, while day-job developers
are more likely to produce bugs. % why do we know this?
A possible cause is that day-job developers are required to make changes,
while the daily developers are motivated purely by interest, and
unlikely to be pressured to fix bugs on any particular schedule.

% we need to say what the two bars mean in the figure.

%% \begin{figure}
%% \begin{center}
%% \includegraphics[width=0.45\textwidth]{linux_day_per_experience.png}
%% \end{center}
%% \caption{Linux percentage of bug introductions and percentage of total commits per author experience}
%% \label{fig-linux-experience}
%% \end{figure}

%% \begin{figure}
%% \begin{center}
%% \includegraphics[width=0.45\textwidth]{firefox_day_per_experience.png}
%% \end{center}
%% \caption{Firefox percentage of bug introductions and percentage of total commits per author experience}
%% \label{fig-firefox-experience}
%% \end{figure}

\paragraph{Developer Experience Classification}
The final head-to-head comparison we performed is based on the amount
of experience per author. We divided them up into 6-month intervals. 
Our results from Linux and Firefox are shown in Figure
\ref{fig-linux-experience} and Figure \ref{fig-firefox-experience}
respectfully. For Linux authors with less than 2 years of experience
are more likely to commit a bug while after 2 years the likelihood
decreases. However the authors which committed throughout the history
of the project are more likely to commit a bug which may be because
they wrote the majority of the code. For Firefox, again the
developers with less than 6 months of experience are more likely to
commit a bug. The is a spike between 21 and 26 months as well. We
believe these results may not be accurate due to the possiblity of
our experience metric being crude and not representative of the 
actual experience with the project.

%% \begin{figure}
%% \begin{center}
%% \includegraphics[width=0.45\textwidth]{linux_bug_life.png}
%% \end{center}
%% \caption{Linux number of bugs against bug lifetimes in months}
%% \label{fig-linux-buglife}
%% \end{figure}

%% \begin{figure}
%% \begin{center}
%% \includegraphics[width=0.45\textwidth]{firefox_bug_life.png}
%% \end{center}
%% \caption{Firefox number of bugs against bug lifetimes in months}
%% \label{fig-firefox-buglife}
%% \end{figure}

Finally we determined the bug lifetimes for both projects to contrast
them to previous work. We plotted the number of bugs for each lifetime
divided into 4 month intervals. The results for Linux are shown in
Figure \ref{fig-linux-buglife}, the average bug lifetime is 1.39
years. The results for Firefox are shown in Figure
\ref{fig-firefox-buglife}, with an average bug lifetime of 0.97
years. This shows a reduction in the average bug lifetime for Linux (in
2002 the average lifetime was 1.8 years). This indicates that
developers are improving as their development process becomes more
refined. We also see the average lifetime is lower for Firefox, this
may be due to the complexity of the software, size of the software or
the amount of users (the more users and complex, the more likely bugs will be
revealed).

\begin{figure}
\begin{center}
\includegraphics[width=0.45\textwidth]{linux-bugginess-author-class.pdf}
\end{center}
\caption{Linux percentage of buggy commits per author classification}
\label{fig-linux-bugginess-author-class}
\end{figure}

\begin{figure}
\begin{center}
\includegraphics[width=0.45\textwidth]{postgresql-bugginess-author-class.pdf}
\end{center}
\caption{PostgreSQL percentage of buggy commits per author classification}
\label{fig-postgresql-bugginess-author-class}
\end{figure}

\begin{figure}
\begin{center}
\includegraphics[width=0.45\textwidth]{linux-bugginess-experience.pdf}
\end{center}
\caption{Linux percentage of buggy commits per author experience}
\label{fig-linux-bugginess-experience}
\end{figure}

\begin{figure}
\begin{center}
\includegraphics[width=0.45\textwidth]{postgresql-bugginess-experience.pdf}
\end{center}
\caption{PostgreSQL percentage of buggy commits per author experience}
\label{fig-postgresql-bugginess-experience}
\end{figure}


\begin{figure}
\begin{center}
\includegraphics[width=0.45\textwidth]{linux-introductions-day.pdf}
\end{center}
\caption{Linux introductions per commit per day}
\label{fig-linux-introductions-day}
\end{figure}

\begin{figure}
\begin{center}
\includegraphics[width=0.45\textwidth]{postgresql-introductions-day.pdf}
\end{center}
\caption{PostgreSQL introductions per commit per day}
\label{fig-postgresql-introductions-day}
\end{figure}

\begin{figure}
\begin{center}
\includegraphics[width=0.45\textwidth]{linux-severity-day.pdf}
\end{center}
\caption{Linux severity of changes per day}
\label{fig-linux-severity-day}
\end{figure}

\begin{figure}
\begin{center}
\includegraphics[width=0.45\textwidth]{postgresql-severity-day.pdf}
\end{center}
\caption{PostgreSQL severity of changes per day}
\label{fig-postgresql-severity-day}
\end{figure}

\begin{figure}
\begin{center}
\includegraphics[width=0.45\textwidth]{linux-introductions-hour.pdf}
\end{center}
\caption{Linux introductions per commit per hour}
\label{fig-linux-introductions-hour}
\end{figure}

\begin{figure}
\begin{center}
\includegraphics[width=0.45\textwidth]{postgresql-introductions-hour.pdf}
\end{center}
\caption{PostgreSQL introductions per commit per hour}
\label{fig-postgresql-introductions-hour}
\end{figure}

\begin{figure}
\begin{center}
\includegraphics[width=0.45\textwidth]{linux-severity-hour.pdf}
\end{center}
\caption{Linux severity of changes per hour}
\label{fig-linux-severity-hour}
\end{figure}

\begin{figure}
\begin{center}
\includegraphics[width=0.45\textwidth]{postgresql-severity-hour.pdf}
\end{center}
\caption{PostgreSQL severity of changes per hour}
\label{fig-postgresql-severity-hour}
\end{figure}

\begin{figure}
\begin{center}
\includegraphics[width=0.45\textwidth]{linux-bugginess-day.pdf}
\end{center}
\caption{Linux percentage of buggy commits per day}
\label{fig-linux-bugginess-day}
\end{figure}

\begin{figure}
\begin{center}
\includegraphics[width=0.45\textwidth]{postgresql-bugginess-day.pdf}
\end{center}
\caption{PostgreSQL percentage of buggy commits per day}
\label{fig-postgresql-bugginess-day}
\end{figure}

\subsection{Day-of-week Results} Having determined our false positive rate, 
we continue by presenting the results of our analysis.  First, we
determine if the day of the week has any impact on the likelihood of
producing bugs. For the following graphs, the percentage of
introduction commits indicates the amount of commits which are bug
introductions, as a propostion of the total number of
commits. Figure~\ref{fig-linux-bugginess-day} presents our results for
Linux. We see that Saturday is the worst day, followed by Thursday,
while Monday has the fewest percentage of bug introductions. The
results for Firefox are shown in Figure~\ref{fig-postgresql-bugginess-day} and
agree with Thursday being one of the worst days, while Monday is the
best day for committing changes which do not introduce bugs. A
possible explanation is that developers rest over the weekend and have
ample time to think about the problem before coding a solution on
Monday, when they know exactly what to do. Per-day bugginess may also
be correlated with the size of the change, which we plan to investigate
later, as an addition to our technique.


\subsection{Validation} 
To validate our results, we first observed the types of commits from both Linux
and Firefox and classified changes related to these bugs.
% more detail!
Table~\ref{tbl-changes} presents our results. A major difference
between Linux and Firefox is in the ``none'' row, where Linux has
7.7\% of changes while Firefox has 43\% of changes. We believe that
this is because our analysis does not consider changes to JavaScript,
which is an important part of Firefox; these changes also appear to
actually be ``modify'' changes, which differs by a similar amount
in the other direction.

% maybe move this to the results section
We only considered patches and bugs that are introduced after April 16th, 2005, when
the Linux kernel started to use Git, because the patches before this date do not have 
time information that is accurate to the hour.

% we need to have a more detailed story above. The other numbers
% aren't actually similar either. -PL


%% We also noted that approximately 7.7\% of the bug fixes for Linux were
%% purely for comments, indicating that developers spend a non-trivial
%% amount of time on comments. For Linux, approximately 51\% of the
%% changes do not include additions. We therefore manually checked (a
%% subsample of) the 49\% more difficult cases to evaluate the
%% effectiveness of our heuristic for additions.

For Linux, we manually checked a random sample of 50 bugs: 11
additions, 16 additions and modifications, 6 additions and removals,
and 17 with all types of changes. We found 10 of the 50 bugs to be
false positives. Similarly, for Firefox, we randomly sampled 50 bugs:
9 additions, 3 additions and modifications, 18 additions and removals,
and 20 with all types of changes. We found 13 of the 50 to be false
positives. 

We next describe our false positives, which included three rare cases
and two common cases. The rare cases included: 1) our analysis
classified a commit message for fixing a merge commit as describing a
fix. 2) We found an author incorrectly blamed for a bug: the bug arose
from a check which was only required, and inadvertently omitted, from
a later revision. 3) The removal of dead code triggered an (invalid)
bug report. More commonly, we found that sometimes 4) a change was
reverted but re-added in a later version; and 5) refactoring changes,
which moved or renamed functions. One might argue that such changes
could actually be bug fixes.

% we didn't say what blame means.
 
Our false positive rate has an upper bound between 20\%-26\%. We
believe this is an upper bound since we manually checked the more
difficult bug-fixing cases. Therefore our data is representative.

