% This is "sig-alternate.tex" V1.9 April 2009
% This file should be compiled with V2.4 of "sig-alternate.cls" April 2009
%
% ----------------------------------------------------------------------------------------------------------------
% This .tex file (and associated .cls V2.4) produces:
%       1) The Permission Statement
%       2) The Conference (location) Info information
%       3) The Copyright Line with ACM data
%       4) NO page numbers
%
% as against the acm_proc_article-sp.cls file which
% DOES NOT produce 1) thru' 3) above.
%
% Using 'sig-alternate.cls' you have control, however, from within
% the source .tex file, over both the CopyrightYear
% (defaulted to 200X) and the ACM Copyright Data
% (defaulted to X-XXXXX-XX-X/XX/XX).
% e.g.
% \CopyrightYear{2007} will cause 2007 to appear in the copyright line.
% \crdata{0-12345-67-8/90/12} will cause 0-12345-67-8/90/12 to appear in the copyright line.
%
%
% For tracking purposes - this is V1.9 - April 2009

\documentclass{sig-alternate}

\clubpenalty=300 %10000
\widowpenalty=300 %10000
%\brokenpenalty=1000 %10000

\usepackage{times}
\usepackage{url}

%------------------------------------------------------------------------- 
% take the % away on next line to produce the final camera-ready version 
%\pagestyle{empty}

%------------------------------------------------------------------------- 
\usepackage{arydshln}
\usepackage{boxedminipage}
\usepackage{xspace}
\usepackage{listings}

%Squeeze space in bibliographybg
  \let\oldthebibliography=\thebibliography
  \let\endoldthebibliography=\endthebibliography
  \renewenvironment{thebibliography}[1]{%
    \begin{oldthebibliography}{#1}%
      \setlength{\parskip}{0ex}%
      \setlength{\itemsep}{0ex}%
  }%
  {%
    \end{oldthebibliography}%
  }

\lstset{language=C}

\newcommand{\code}{\tt \small}

\newcommand{\comment}[1]{{}}
%\newcommand{\bfc}[1]{bug-fixing commits}
%\newcommand{\bic}[1]{bug-introduction commits}
\newcommand{\note}[1]{{\bf *** NOTE: {#1} ***}}


\begin{document}
%
% --- Author Metadata here ---
\conferenceinfo{MSR}{'11 Honolulu, Hawaii, USA}
%\CopyrightYear{2007} % Allows default copyright year (20XX) to be over-ridden - IF NEED BE.
%\crdata{0-12345-67-8/90/01}  % Allows default copyright data (0-89791-88-6/97/05) to be over-ridden - IF NEED BE.
% --- End of Author Metadata ---

\title{Do time of day and developer experience affect commit bugginess?}
%% \subtitle{Subtitle Text, if any}


%\title{Alternate {\ttlit ACM} SIG Proceedings Paper in LaTeX
%Format\titlenote{(Produces the permission block, and
%copyright information). For use with
%SIG-ALTERNATE.CLS. Supported by ACM.}}
%\subtitle{[Extended Abstract]
%\titlenote{A full version of this paper is available as
%%\textit{Author's Guide to Preparing ACM SIG Proceedings Using
%\LaTeX$2_\epsilon$\ and BibTeX} at
%\texttt{www.acm.org/eaddress.htm}}}
%
% You need the command \numberofauthors to handle the 'placement
% and alignment' of the authors beneath the title.
%
% For aesthetic reasons, we recommend 'three authors at a time'
% i.e. three 'name/affiliation blocks' be placed beneath the title.
%
% NOTE: You are NOT restricted in how many 'rows' of
% "name/affiliations" may appear. We just ask that you restrict
% the number of 'columns' to three.
%
% Because of the available 'opening page real-estate'
% we ask you to refrain from putting more than six authors
% (two rows with three columns) beneath the article title.
% More than six makes the first-page appear very cluttered indeed.
%
% Use the \alignauthor commands to handle the names
% and affiliations for an 'aesthetic maximum' of six authors.
% Add names, affiliations, addresses for
% the seventh etc. author(s) as the argument for the
% \additionalauthors command.
% These 'additional authors' will be output/set for you
% without further effort on your part as the last section in
% the body of your article BEFORE References or any Appendices.

\numberofauthors{1} %  in this sample file, there are a *total*
% of EIGHT authors. SIX appear on the 'first-page' (for formatting
% reasons) and the remaining two appear in the \additionalauthors section.
%
\author{
% You can go ahead and credit any number of authors here,
% e.g. one 'row of three' or two rows (consisting of one row of three
% and a second row of one, two or three).
%
% The command \alignauthor (no curly braces needed) should
% precede each author name, affiliation/snail-mail address and
% e-mail address. Additionally, tag each line of
% affiliation/address with \affaddr, and tag the
% e-mail address with \email.
%
% 1st. author
\alignauthor
Jon Eyolfson, Lin Tan, Patrick Lam\\ 
       \affaddr{University of Waterloo} \\
       \affaddr{200 University Avenue West}\\
       \affaddr{Waterloo, ON N2L3G1, Canada} \\ 
       \affaddr{\em jon@eyolfson.com, lintan@uwaterloo.ca, p.lam@ece.uwaterloo.ca}
}



\maketitle

%% \category{CR-number}{subcategory}{third-level}

%% \terms
%% term1, term2

%% \keywords
%% runtime monitoring, verification, tracematches, dynamic binary translation

\maketitle

\begin{abstract}
There are many factors which may contribute to patch quality, our goal
is to find these correlations. Previous studies used day of the week
in order to find correlations. We extend this by using time of the day
and author frequency/experience. We discuss the impact of our work
through motivating examples, which may lead to higher quality code.

We implemented a tool for software repositories which determines
bug-fixing and bug-introducing commits. We describe our implementation
and evaluate it on open source projects. Finally, we present our
findings from these projects.

%% If we need more information...
%%
%% First, given a software repository, we extract bug-fixing commits
%% using a keyword search. Then, we automatically determine the
%% bug-introduction commits from this using ``blame''
%% information. Finally, we store this in a database and analyze the
%% commit and author information.

%% \comment{In this paper we attempt to find a correlation between the commit time
%% of a code change and the rate of introduction of bugs. In other words,
%% are developers more prone to bug-inducing changes during certain hours
%% of the day or not. Before we start our search we were expecting our
%% results to show that before lunch hour, in anticipation of going for
%% lunch, and towards the end of the working day in anticipation of going
%% home. Previous studies have explored whether there is a correlation
%% between the bug-inducing changes rate and the day of the week, various
%% mining techniques of bug-inducing changes, but to our knowledge no
%% work has been to try and find a specific time slots during the day
%% when developers might be more likely to introduce bugs. We mined data
%% from Linux and Firefox using their software repositories. We developed
%% an automated method for extracting information which finds a patch
%% containing a fix and links it to patches which introduced the bug,
%% which required the fix. We found the false positive rate to be under
%% 20\% after randomly sampling 100 reports. Our study has found that Thursday is a bad day to code, while
%% Mondays are surprisingly good. Also between 12AM and 9AM, developers are more prone
%% to introducing errors, while the least amount of bugs are introduced between
%% 11AM and 3PM. Our data also shows that daily committers are less prone to introducing
%% bugs, while day-job users are more prone. Finally, the bug lifetimes seem to decay
%% exponentially, and on average is longer for larger projects.}
\end{abstract}



% There's nothing stopping you putting the seventh, eighth, etc.
% author on the opening page (as the 'third row') but we ask,
% for aesthetic reasons that you place these 'additional authors'
% in the \additional authors block, viz.
\additionalauthors{}
%\date{30 July 1999}

% Just remember to make sure that the TOTAL number of authors
% is the number that will appear on the first page PLUS the
% number that will appear in the \additionalauthors section.


% A category with the (minimum) three required fields
%\category{}{}{}[]
%\category{}{}{}[]

%\category{H.4}{Information Systems Applications}{Miscellaneous}
%A category including the fourth, optional field follows...
%\category{D.2.8}{Software Engineering}{Metrics}[complexity measures, performance measures]

%\terms{Theory}

%\keywords{Empirical Study, Bug detection}


%------------------------------------------------------------------------------



\input intro
\input methods
%\input idea % idea, design, and methods
\input results
\input related
\input conclusions



%\section{Introduction}

Software demands high reliability. However, as its complexity increases, 
modern software inevitably contain bugs, which hurt software reliability.
The modern software development paradigm further complicates the situation: 
Many modern software, e.g., Firefox, the Linux kernel, Apache, and Eclipse, 
are developed by hundreds of or thousands of developers over decades~\cite{ccomment} 
in a distributed manner. Developers with different programming experiences, 
different time commitments, different programming styles, from all over the world,
work on the same software project at different times. They join and leave a project 
from time to time over decades. 

Therefore, code developed in the modern paradigm has different social characteristics 
that are beyond the code itself. For example, different code segments in the same
software project may written by developers with  
varies level of expertise, different time 
commitment, different working hours, diverse culture, etc.
The code may be checked in at different times in different time zones. 

Such social characteristics provide a rich and unique source of information for us to 
understand software bugs. Just to name one example, it would be beneficial to know if  
commit time have implications on the qualify of the commit. 
Are commits submitted after midnight more buggy than other commits? 
Such correlation results may be useful to predict what commits are more likely
to be buggy so that we can budget more testing effort on these commits, just
like prior studies~\cite{graves00predicting, guo04robust, ostrand05predicting},
which predict bugs based on code complexity, 
code locations, amount of in-house testing, etc. 
Another interesting question is, whether more experienced developers are more or less
likely to introduce buggy commits. 

In this paper, we study the social characteristics of modern software to understand 
the correlation between these social characteristics with the correctness of the software.  
Specifically, we study the latest version of the Linux kernel and XXX with XXX-XXX millions lines 
of code, XXX-XXX versions, developed over XXX-XXX years. We study ...

There are two main challenges in studying software social characteristics.
First, it is challenging to know if a commit is buggy or not,
% A challenge for studying the ``bugginess'' of commits is
%that it is hard to know whether a commit is correct or not at the time of commit, 
as bugs in the commit are often not 
discovered or reported until days, months, or even years after the commit. 
We use a reverse chronological method to address this challenge. Specifically, we 
start from bug fixes (commits that fix software bugs), and then use the version 
control system to understand when the corresponding bugs were introducted. 
In other words, we extract the bug introduction time for bug fixes 
from the software version control systems. Then we can know 
collectively when bugs are more likely to be introduced.  
%In particular, we leverage the ``git blame'' functionality to identify the last modification time of each buggy line.

Additionally, it is difficult to know how experienced the developers are in an open source development
environment as such information is not documented and subjective. 
We make our best-effort approximation by calculating the length and frequency of a developer
contributing to an open source project, which allows us to discover important findings 
and implications regarding developers' experience level and code correctness. 

In addition to the social characteristics, we also study several other commit characteristics, such as
the commit sizes, XXX, XXX, etc. 
Our major findings 
%that can benefit language/tool designers and system programmers 
include ($\S$ denotes in which
section the finding and its {\em implications} are discussed):

\begin{list}{\labelitemi}{\topsep=0pt\parsep=0pt\leftmargin=9pt\itemindent=0pt}

\vspace{0.05in}
  \item {\bf Finding 1 ($\S$\ref{sec:finding1}):} 
Commits that are checked into the softare repository between XXX-XXX AM  are  XXX\% more likely
to be incorrect. 
The result indicate that developers may want to double check the code they write  
before they commit between these hours. It may also to beneficial for the version control
system to warn the developers of late night commits to improve software reliability. 

\item {\bf Finding 2 ($\S$\ref{sec:finding2}):} 
\item {\bf Finding 3 ($\S$\ref{sec:finding3}):} 
\end{list}


%Developers spend a large amount of time
%diagnosing bugs and fixing bugs by applying patches over the buggy software versions. 
%As modern software are often very complex and developed by thousands of developers 
%distributedly over many years, many of the patches are not correct, e.g., not fixing
%the bug, or introducing new bugs, etc~\cite{}. A recent study shows that XXX\% of patches
%are buggy. 

\comment{
The area of bug detection and resolution is currently a very active area of research
in the software related disciplines. The process of detecting and resolving software
bugs is an extremely costly process because of the manpower required to perform
quality assurance on the developed software and having developers perform the fixes.
It is estimated that 20\% - 30\% of the time spent on a software project is spent on
testing and integration \cite{2004-industry}. In addition to the cost of manpower,
the time required to complete these tasks can be costly on the release date of the
software. As such, for the majority of software companies, the process of quality
assurance is costly and time consuming. To put the cost into perspective, a study
conducted by the National Institute of Standard and Technology in 2002 reported that
software bugs cost the economy of the United States approximately \$59.5 billion
annually \cite{2002-economic}. The study also showed that approximately 64\% of
the costs of software bug fixes are incurred by the end users, making this issue
of extreme importance for the consumers as well \cite{2002-economic}.

In our paper we will show the data collected from two large projects, which
are complex and representative of most software projects, namely Linux and Firefox.
Our data includes: percentage of bugs introduced vs total percentage of commits for
every day of the week, percentage of bugs introduced every hour vs percentage of total commits
every hour, percentage of introduction vs percentage of total commits by author classification,
percentage of bugs introduced vs percentage of total commits grouped by author experience,
and bug lifetime.

Our paper does not focus on solving a problem, but instead tries to find correlations between
bug introductions and days of the week, hours of the day, commit introductions by author
experience, commit introductions by author classification, and bug life time. This information
can be used to try and reveal any potential correlations that could help leaders
in the software industry determine ways to minimize any potential effects
of working on a certain day, or during a certain hour. Collecting and analyzing
data with regards to bug introduction will surely be beneficial to reducing bug
introduction rates by examining some of the potential causes.

This paper will be organized in the following fashion: Section
\ref{sec-idea} will give an insight to our idea and how we approached
the problem, Section \ref{sec-impl} discusses our implementation for
automatically finding bug introduction points, Section
\ref{sec-results} presents the results we found, Section
\ref{sec-related} will discuss the related academic work which our
work compliments, Section \ref{sec-future} explores possible future
work, and finally Section \ref{sec-conclusion} concludes our work.
}

%\input{design}
%- find the commits which fixed a bug
- a bug is an _
- work backwards to find the commits which introduced the bug (ignore merge commits)
 - using the diff information
 - talk about additions/removals/modifications
 - blame
- talk about which type of data we collect
 - author information
 - commit information which

Our goal is to find commits which fix a bug, along with the commits
which introduced that bug. First, we detect a bug-fixing commit $c$ by
searching the commit messages for common keywords such as ``fix'' and
``bug''. Next, we work backwards to locate the commits that introduced
the bug fixed in $c$. Using commit $c$ as a starting point, we compute
a diff for each file between the version after $c$ and the previous
version of that file, and record the line numbers changed in
$c$. Finally, we use the repository metadata to find the commit(s)
$c'$ which were responsible for the previous versions of the files
which were fixed, and we conclude that the commits $c'$ introduced the
bug.

Note that any change in $c$ which removes or modifies an existing line
of code is easy to attribute to a previous commit $c'$, since the
affected line of code existed in $c'$. However, a change in $c$ which
adds a new line of code has no corresponding change in any previous
revision, since this line did not previously exist. In that case, we
attribute responsibility to the commit which introduced the line just
before the new line.

\paragraph{Data collected.}
Executing the above algorithm gives us data about each commit, as well
as the authors of the commits\footnote{Name/email.}.  We record the
following data for each commit: author, time, number of lines changed,
and number of times the commit introduced a bug later corrected. For
each author, we record the name, email(s), experience and
classification. We define the bug lifetime as the time from the
earliest commit which introduced a bug to the bug-fixing commit.

We compute an author's experience and classification based on the
frequency and duration of that author's commits to a project. Author
experience counts the time between an author's earliest and latest
commits. Author classification describes the author's most-common
frequency between consecutive commits: daily, weekly, monthly, and
single. Note that, for author classification purposes, we ignore
consecutive commits within 30 minutes of each other. As a sub-class of
the daily commiter classification, we also use a heuristic to identify
commiters who worked on the repository as part of their day job,
namely those for which 85\% of commits are between 8 AM and 4 PM
Monday to Friday.

\paragraph{Validation.}
To validate our data we randomly sample commits, manually verify that
the commit is a bug-fix and that the correct commits introduced the
bug. We focused our manual efforts on changes which included
additions, as conceptually, they are the largest threat to validity.
% Currently there are no false negative rates, we probably have time to sample non fix commits

* time zone information

%\section{Results}
\label{sec-results}

In this section, we present the results obtained from carrying out our
methodology, and discuss some of the implications of our results. Most of our
results investigate the effect of an independent variable (time-of-day and
developer experience/frequency classifications) on the likelihood of a commit to
be a bug-introducing commit, or \emph{bugginess}. We also describe our findings
with respect to the day of the week, which allows us to compare our results to
those in~\cite{sliwerski-msr-2005}. We also discuss our finding that some
bug-fixing commits only changed comments. Finally, we explain the precision and
recall of our methodology and how we computed these figures.

\subsection{Project Characteristics}
\label{sec-proj-char}

We chose two large open-source software repositories for our investigations:
Linus Torvalds's mainline Linux kernel %, from \url{git.kernel.org} 
and PostgreSQL. %, from the project's repository at \url{git.postgresql.org}.
%
Table~\ref{tab:characteristics} summarizes the characteristics of our
repositories. The row ``lines of code'' refers to the current size of
the code in the repository. The row ``\# bug-introducing'' shows that
23.7--25.5\% of the commits are buggy, which is slightly lower than the
previously reported figure of nearly 40\% for a commercial switching
system~\cite{smallCommits05}. Note that the PostgreSQL repository was carefully
converted from CVS using {\code cvs2git} in September 2010. We discussed the quirks of the PostgreSQL repository in Section~\ref{sec:method}.

\begin{table}
\begin{tabular}{l|r|r}
& {\bf Linux kernel} & {\bf PostgreSQL} \\ \hline
First commit & April 16, 2005 & July 9, 1996 \\
Cloned & November 21, 2010 & January 24, 2011 \\
Lines of code & over 5 million & over 750,000 \\
Number of authors & 8,594 & 34 \\
Number of commits & 222,332 & 31,098 \\
\# bug-introducing & 56,681 (25.5\%) & 7,366 (23.7\%) \\
\# bug-fixing & \linuxBFC & \postBFC
\end{tabular}
\caption{\label{tab:characteristics}Characteristics of the Linux kernel and PostgreSQL repositories.}
\end{table}

%% The Linux repository was cloned on November 21, 2010, from Linus Torvalds's
%% mainline kernel, hosted at \url{git.kernel.org}; this repository contains
%% history back to April 16, 2005. This repository contains 222,332 commits
%% contributed by 8,594 authors. Of these commits, we identified 56,681
%% bug-introducing commits and 57,028 bug-fixing commits. The tip of the
%% repository contains over 5 million lines of code.

%% The PostgreSQL repository was cloned on January 24, 2011, from the project's
%% repository at \url{git.postgresql.org}; it contains history to July 9, 1996,
%% translated from CVS using \code{cvs2git}. This repository contains 31,098
%% commits contributed by 34 authors. We identified 7,366 bug-introducing commits
%% and 4,399 bug-fixing commits. The tip of the repository contains over 750,000
%% lines of code.

\begin{figure*}[tbh]
\centering
\subfigure[{Linux kernel}]{
 \label{fig-linux-bugginess-hour}
 \includegraphics[width=\columnwidth]{linux-bugginess-hour.pdf}
}
\subfigure[PostgreSQL]{
 \label{fig-postgresql-bugginess-hour}
 \includegraphics[width=\columnwidth]{postgresql-bugginess-hour.pdf}
}
\caption{\label{fig-bugginess-hour}Percentage of buggy commits (bars) and total number of commits (circles) versus time-of-day}
\end{figure*}

\begin{figure*}[tbh]
\centering
\subfigure[{Linux kernel}]{
 \label{fig-linux-introduction-hour}
 \includegraphics[width=\columnwidth]{linux-introductions-hour.pdf}
}
\subfigure[PostgreSQL]{
 \label{fig-postgresql-introduction-hour}
 \includegraphics[width=\columnwidth]{postgresql-introductions-hour.pdf}
}
\caption{\label{fig-introduction-hour}Subsequent bug fixes per commit (bars) and
 total commits (circles) versus time-of-day}
\end{figure*}

%% \begin{figure*}[tbh]
%% \centering
%% \subfigure[{Linux kernel}]{
%% \label{fig-linux-severity-hour}
%% \includegraphics[width=\columnwidth]{linux-severity-hour.pdf}
%% }
%% \subfigure[PostgreSQL]{
%%  \label{fig-postgresql-severity-hour}
%%  \includegraphics[width=\columnwidth]{postgresql-severity-hour.pdf}
%% }
%% \caption{\label{fig-severity-hour}Severity of changes (bars) and average number
%%  of lines per commit (circles) versus time-of-day}
%% \end{figure*}

\subsection{Time-of-day} 
\label{sec-time-of-day}

Figure~\ref{fig-bugginess-hour} presents our results correlating the time-of-day
of a commit with its bugginess. The graphs compare the time-of-day of each
commit, in the committer's local time on a 24-hour clock, to the percentage of
bug-introducing commits. The solid horizontal line
indicates the overall percentage of buggy commits in each project; bars
shorter than the line indicate that commits at that hour were less likely to be
buggy, while bars taller than the line indicate hours with more-buggy
commits. The graphs also contain the raw number of commits at each hour, 
indicated by circles.

%% While our principal metrics are the absolute count and percentage of
%% bug-introducing commits, we also present results of a third metric, \emph{bug
%% severity}. This metric attributes less weight to a large commit which
%% introduces a small number of bugs relative to its size, and more weight to small
%% commits which introduce relatively large numbers of bugs. We define bug severity
%% to be the number of bug introductions per changed non-comment lines of
%% code. Figure~\ref{fig-severity-hour} presents our data correlating severity of
%% changes, along with average commit size, with time-of-day.

Figure~\ref{fig-bugginess-hour}, which summarizes bugginess percentages, shows a
noticeable increase in the amount of commits which introduce a bug between 00:00
(midnight) and 04:00 (4 AM). After 04:00, commits tend to be less buggy than
average, gradually increasing until noon. In the Linux kernel, commits between
noon and midnight fluctuate around the average bugginess level, while the
PostgreSQL commits are generally above the average bugginess level between 16:00
(4 PM) and 20:00 (8 PM), and then below the average bugginess level between 20:00
(8 PM) and 00:00 (midnight). Figure~\ref{fig-introduction-hour}, which shows the
number of subsequent bug-fixing commits for each bug-introducing commit 
(indicating how difficult it was to correct a bug), follows the trends from
Figure~\ref{fig-bugginess-hour}. Note that even the smallest total number of
commits, for any hour, is 139 for PostgreSQL (and an order of magnitude higher
for the Linux kernel), so that all of the depicted bug introduction rates are
meaningful.

%% Figures~\ref{fig-linux-severity-hour} and~\ref{fig-postgresql-severity-hour}
%% show that the severity of late-night/early-morning changes (before 9 AM) is
%% surprisingly high, while there is no general trend in the average commit size.
%% The average size of Linux commits fluctuates quite a lot; the sizes of
%% PostgreSQL commits are more stable. For PostgreSQL, developers tend to commit
%% slightly smaller changes between 5 AM and noon, with a spike at 8 AM.

We also investigated correlations between the time-of-day and the number of
bug-fixing commits, rather than the bug-introducing commits that we showed
above. The proportion of total commits that are bug-fixing commits
stayed almost constant, independent of the hour; the graphs (not shown) have
exactly the same shape as that of the circles in
Figure~\ref{fig-bugginess-hour}. This suggests that the fact that a commit is
bug-fixing is independent of its other characteristics.

\begin{table}[tbh!]
\begin{center}
\small
\begin{tabular}{r|r|r}
\multicolumn{1}{c}{} & \multicolumn{2}{c}{{\bf P-value}} \\
\multicolumn{1}{c|}{{\bf Hour}} & \multicolumn{1}{c|}{{\bf Linux kernel}} &
\multicolumn{1}{c}{{\bf PostgreSQL}} \\
\hline
0  & 9.62E-18 & 0.245   \\
1  & 4.59E-18 & 0.00205 \\
2  & 1.62E-19 & 0.00748 \\
3  & 0.197    & 0.0382  \\
4  & 0.232    & 0.0348  \\
5  & 0.173    & 0.133   \\
6  & 0.0116   & 0.464   \\
7  & 1.56E-15 & 0.308   \\
8  & 2.62E-26 & 0.0494  \\
9  & 4.54E-20 & 3.80E-6 \\
10 & 3.25E-11 & 0.00108 \\
11 & 6.88E-6  & 0.00179 \\
12 & 6.13E-7  & 2.63E-4 \\
13 & 0.0255   & 0.258   \\
14 & 0.00447  & 0.114   \\
15 & 0.366    & 0.386   \\
16 & 0.436    & 2.40E-5 \\
17 & 0.00929  & 0.0456  \\
18 & 0.301    & 0.00176 \\
19 & 0.471    & 2.70E-4 \\
20 & 0.0695   & 0.00314 \\
21 & 4.91E-6  & 0.311   \\
22 & 0.00115  & 0.00433 \\
23 & 2.42E-4  & 0.0509  \\
\end{tabular}
\end{center}
\caption{\label{tbl-pvalues}Linux kernel and PostgreSQL bugginess p-values}
\end{table}

Table~\ref{tbl-pvalues} presents p-values evaluating the
statistical significance of the per-hour commit bugginess for
Linux and PostgreSQL.  The null hypothesis is that each
hour has the same probability as the overall bugginess for each
project.  Typically, a p-value less than 0.05 indicates that the null
hypothesis is rejected, and the corresponding result is considered to
be statistically significant.  Therefore, our p-value results show
that the differences in bugginess of different hours are statistically
significant. Concretely, the p-values allow us to conclude that
commits introduced between 00:00 (midnight) and 04:00 (4 AM) are
buggier than average with statistical significance.

%We only calculated p-values for bugginess since it is discrete, 
%an individual commit is either buggy or not.

\newpage
\paragraph{Discussion}

Code does not spontaneously improve if left to ``mature'' for 4 hours; 
our results do not indicate causation,
but instead demonstrate a correlation between code committed early in the 
morning and increased bugginess. We do not speculate about the cause
of this correlation; however, the results in Section~\ref{sec:toddev-exp}
imply that this correlation holds for both inexperienced and experienced
developers.
Our results do suggest that developers, being aware of such a 
correlation, may want to double-check code before performing
late-night commits (midnight--4:00 AM). It may also be
beneficial for version control systems or IDEs to warn developers about
the perils of late-night
commits. 
Our p-values indicate that our
observed bugginess differences between late-night/early-morning and all other commits are statistically significant.

Our results also suggest that tired developers (midnight--4 AM) are more likely to
miss corner cases in a pre-commit review (for PostgreSQL) or while finalizing
their patch (for the Linux kernel). Furthermore, we can observe that commits
before noon are least likely to be bug-introducing; perhaps committers are most
careful in those hours.

\begin{figure}[t!hb]
\begin{center}
\includegraphics[width=\columnwidth]{linux-bugginess-author-class.pdf}
\end{center}
\caption{\label{fig-linux-bugginess-author-class}Linux percentage of buggy
 commits (bars) and number of commits (circles) versus author classification}
\end{figure}

\subsection{Developer Characteristics}
\label{sec-dev-char}

We next present our findings with respect to developers' commit frequency and
experience. Developers' commit frequency summarizes the frequency
of a developer's contributions to a project, while developer experience tracks
how long a developer has contributed to a particular project.

\subsubsection{Commit Frequency Classification} 

As we described in Section~\ref{sec:data}, one of the ways that we classify
developers is according to frequency, i.e. most-common interval between
consecutive commits---daily, weekly, monthly, other, or single. This
information is only interesting for the Linux kernel, as almost all (28/34) of
PostgreSQL's committers are daily. We computed the bugginess rates for each of
these classes of developers and plot author classification versus
bug-introduction percentage in
Figure~\ref{fig-linux-bugginess-author-class}. The graph also presents the
number of commits by each class. Note that the Linux kernel has 49 day job
authors, who provide quite a few of the total commits, 801 daily authors, who
account for the overwhelming majority of commits, 238 weekly, 288 monthly, 3562
other (less than 20 commits and more than 1 commit), and 3664 single-commit authors.

Our results show that the Linux kernel developers who commit changes daily, but
not as their day job, produce the largest number of commits and the smallest
number of bug-introducing commits, followed by the single-commit authors (whose
patches would presumably be simple or closely-reviewed). The day job, weekly,
and monthly committers all produce slightly more bug-introducing commits than
average.

\paragraph{Discussion}

A possible cause for the difference between day-job and daily committers is that
day-job developers might be required to make changes by their employers, while
the daily developers are motivated purely by interest, and unlikely to be
pressured to fix bugs on any particular schedule.

\begin{figure*}[tbh]
\centering
\subfigure[{Linux kernel}]{
 \label{fig-linux-bugginess-experience}
 \includegraphics[width=\columnwidth]{linux-bugginess-experience.pdf}
}
\subfigure[PostgreSQL]{
 \label{fig-postgresql-bugginess-experience}
 \includegraphics[width=\columnwidth]{postgresql-bugginess-experience.pdf}
}
\caption{\label{fig-bugginess-experience}Percentage of buggy commits (bars) and total number of commits (circles) versus author experience}
\end{figure*}

\subsubsection{Developer Experience}
\label{sec-dev-exp}

Figure~\ref{fig-bugginess-experience} compares author experience at time of
commit to the bugginess of the commit. It also presents the total number of
commits by author experience. Note that a plurality of Linux commits are by
authors with fewer than 120 days of experience. Both the Linux and PostgreSQL
data show that bugginess decreases with increased author experience. For Linux,
authors with at least 960 days of experience tend to produce commits that are
less buggy than average, while the similar point for PostgreSQL occurs at 3000
days. The PostgreSQL data also shows a spike at the right, which implies
surprisingly high bugginess in code recently committed by the original authors.

\paragraph{Discussion}

Our data shows that, in general, the more experienced the developers are, the
less likely that their commits are buggy. Without further data, this
correlation does not prove that the developer experience caused more experienced
programmers' commits to be less buggy. While we believe the above causation to
hold, other interpretations are possible; perhaps more experienced developers
wrote more complex code, whose bugs are harder to discover and less likely to be
reported. Nonetheless, our results show that, given the fact that a commit is
from a more experienced developer, one can be more confident about the
correctness of the commit. Such a correlation could be exploited to help predict
buggy code locations.

One can observe a decline in the total number of commits with experience.
We believe that this is due to our sliding scale for author experience.
Consider an author who has committed for 5 years. His or her commits do not
show up in a single circle at the 1800-day mark; instead, they are distributed
throughout the 5 years of the commits, so that a commit on the author's second
birthday gets reported as a commit at day 700. One would therefore expect
more commits from ``inexperienced'' developers, since all developers go through
an inexperienced phase, while only a small number of developers reach the
more experienced phase.

We do not understand the spike in percentage of buggy commits to the
right of the PostgreSQL graph. Possible reasons include the shift to
Git, which is a known historical event, or, more speculatively,
perhaps PostgreSQL recently undertook major ongoing architectural
revisions, carried out by the experienced developers.

\begin{figure*}[tbh]
\includegraphics[width=\textwidth]{linux-bugginess-hour-experienced.pdf}
\caption{\label{fig-linux-bugginess-experienced}Percentage of buggy commits
 (bars) and total number of commits (circles/triangles) versus time-of-day for
 inexperienced and experienced Linux kernel developers}
\end{figure*}

\subsection{Combined Time-of-day and Experience}
\label{sec:toddev-exp}

Figure~\ref{fig-linux-bugginess-experienced} combines data from
Section~\ref{sec-time-of-day} and~\ref{sec-dev-exp} and correlates time-of-day
with commit bugginess for inexperienced and experienced developers,
plotted separately, for Linux. We used
a cutoff of 2 years to separate inexperienced and experienced developers; 
this cutoff divides the number of commits into two approximately-equal groups. 
Horizontal lines in the figure represent overall
bugginess. 

We can see that inexperienced developers tend to do more commits between
midnight and 2 AM than experienced developers, who do more commits between 8 AM
and 4 PM. However, there is a common trend for both; late night commits
(especially between midnight and 2 AM) are more buggy and early morning commits
(between 6 AM and noon) are less buggy.

\paragraph{Discussion}

This result suggests that the correlation between time-of-day versus
bugginess is independent of experience for Linux developers; it occurs
for both inexperienced and experienced developers. It also
shows that experienced developers are much less likely to commit a
bug; the average bugginess for experienced Linux developers is around
21\%, versus to 30\% for inexperienced developers. 

\begin{figure*}[tbh]
\centering
\subfigure[{Linux kernel}]{
 \label{fig-linux-bugginess-day}
 \includegraphics[width=\columnwidth]{linux-bugginess-day.pdf}
}
\subfigure[PostgreSQL]{
 \label{fig-postgresql-bugginess-day}
 \includegraphics[width=\columnwidth]{postgresql-bugginess-day.pdf}
}
\caption{\label{fig-bugginess-day}Percentage of buggy commits (bars) and total
 commits (circles) versus day-of-week}
\end{figure*}

\begin{figure*}[tbh]
\centering
\subfigure[{Linux kernel}]{
 \label{fig-linux-introduction-day}
 \includegraphics[width=\columnwidth]{linux-introductions-day.pdf}
}
\subfigure[PostgreSQL]{
 \label{fig-postgresql-introduction-day}
 \includegraphics[width=\columnwidth]{postgresql-introductions-day.pdf}
}
\caption{\label{fig-introduction-day}Subsequent bug fixes per commit (bars) and total commits (circles) versus day-of-week}
\end{figure*}

%% \begin{figure*}[tbh]
%% \centering
%% \subfigure[{Linux kernel}]{
%%  \label{fig-linux-severity-day}
%%  \includegraphics[width=\columnwidth]{linux-severity-day.pdf}
%% }
%% \subfigure[PostgreSQL]{
%%  \label{fig-postgresql-severity-day}
%%  \includegraphics[width=\columnwidth]{postgresql-severity-day.pdf}
%% }
%% \caption{\label{fig-severity-day}Severity of changes (bars) and average number
%%  of lines per commit (circles) versus day-of-week}
%% \end{figure*}

\subsection{Day-of-week}
\label{sec-day-of-week}

Our next experiment attempted to replicate the results
in~\cite{sliwerski-msr-2005}, and correlates the day of the week of a commit with
its bugginess. Figure~\ref{fig-bugginess-day} compares the day of the week with
the bugginess of the commits on that day (bars), and also displays the total
number of commits per day (circles). Here, the solid horizontal line presents
the overall bugginess of all commits to each project.
Figure~\ref{fig-introduction-day} presents the number of subsequent bug-fixing
commits for each bug-introducing commit per day-of-week.%% , while
%% Figure~\ref{fig-severity-day} presents the severity of bugs per day-of-week, and
%% also presents the average sizes of commits.

Our results, which use a disjoint set of repositories from those
in~\cite{sliwerski-msr-2005}, found about the same bugginess and number of
introductions for each day in the Linux kernel repository, with the lowest
bugginess on Sunday and highest on Monday; for the PostgreSQL repository, we
observe a slight decrease in bugginess on Tuesday, and a noticeable increase on
Sunday. These results are statistically significant with a p-value less than
0.05. Note that, for Linux, Saturday and Sunday each have about half as many
commits as the other days of the week (commits peak on Tuesday and steadily
decrease through Friday). For PostgreSQL, commits fluctuate through the days of
the week and decrease to about 70\% of the weekday volume on the weekend. %% The
%% severity results do not show any particular trends, but we found that the
%% average number of lines per commit on Sunday was surprisingly large for both
%% Linux and PostgreSQL.

\paragraph{Discussion}

We found that commits on different days of week have about the same
bugginess, which does not agree with results from the prior study on two
different open source projects~\cite{sliwerski-msr-2005}. 
We also found that the bugginess per
day-of-week for commits varies for different software projects, implying that
bugginess prediction based on day-of-week may need to be calibrated on a 
per-project basis.

\subsection{Bug Lifetimes}
\label{sec-bug-lifetime}

Recall that the bug lifetime is the amount of time elapsed between the
bug-introducing commit and its bug-fixing commit. Figure~\ref{fig-bug-lifetime}
shows bug lifetimes for the Linux kernel and PostgreSQL, 
grouped in 120 day intervals. We found the average bug lifetime for the Linux
kernel is 1.38 years with a standard deviation of 1.35 years. The average bug
lifetime for PostgreSQL is 3.07 years with a standard deviation of 3.19
years. Note that the distribution of bug lifetimes is similar for both projects;
many bugs are fixed within a 120 day period and the overall lifetime appears to
decrease exponentially.

\paragraph{Discussion}

PostgreSQL may have a longer average bug lifetime due to being a smaller, less
complex project with a smaller user base. We found the sources of the long-time
bugs include race conditions, incorrect calculations and rare corner cases; such
cases are intuitively more likely to be found with a larger user base.

\begin{figure*}[tbh]
\centering
\subfigure[{Linux kernel}]{
 \label{fig-linux-bug-lifetime}
 \includegraphics[width=\columnwidth]{linux-bug-lifetime.pdf}
}
\subfigure[PostgreSQL]{
 \label{fig-postgresql-bug-lifetime}
 \includegraphics[width=\columnwidth]{postgresql-bug-lifetime.pdf}
}
\caption{\label{fig-bug-lifetime}Histogram of bug lifetime counts}
\end{figure*}

\subsection{Comment-only Commits}
\label{sec-comment-only}

A surprisingly large number of bug-fixing commits reported 0
changed lines of code. We performed a random sample on 50 of these commits for
each project and found that almost all of them only changed comments in the
source code, which we did not count as changed lines of code. For the Linux
kernel 2.15~$\pm$~0.10\%\footnote{We report the margin of error with 95\%
 confidence level.} of the bug-fixing commits (1,220
commits\footnote{Estimated based on the percentage of comment-only commits and
 the total number of commits.}) were on comments only. We followed the same
procedure for PostgreSQL and found 2.97~$\pm$~0.45\% (about 131 commits) of
bug-fixing commits were on comments only. These hundreds and thousands of
comment-only commits show that developers spend a nontrivial amount of time
purely maintaining the correctness of comments; these numbers don't even
consider the amount of time that developers incidentally update comments along
with the code.

%% We also noted that approximately 7.7\% of the bug fixes for Linux were purely
%% for comments, indicating that developers spend a non-trivial amount of time on
%% comments. For Linux, approximately 51\% of the changes do not include
%% additions. We therefore manually checked (a subsample of) the 49\% more
%% difficult cases to evaluate the effectiveness of our heuristic for additions.



\subsection{Validation} 
\label{sec-validation}

%% \begin{table}
%% \begin{center}
%% \begin{tabular}{rrrr}
%% & & \multicolumn{2}{c}{Predicted} \\
%% & & \multicolumn{1}{|c}{Fix} & \multicolumn{1}{c}{$\neg$Fix} \\ \cline{2-4}
%% \multirow{2}{*}{Actual} & \multicolumn{1}{c|}{Fix} & 48 & 18 \\
%%                         &  \multicolumn{1}{c|}{$\neg$Fix} & 7 & 127 \\
%% \end{tabular}
%% \end{center}
%% \caption{\label{tbl-linux-confusion}Linux kernel confusion matrix}
%% \end{table}

%% \begin{table}
%% \begin{center}
%% \begin{tabular}{rrrr}
%% & & \multicolumn{2}{c}{Predicted} \\
%% & & \multicolumn{1}{|c}{Fix} & \multicolumn{1}{c}{$\neg$Fix} \\ \cline{2-4}
%% \multirow{2}{*}{Actual} & \multicolumn{1}{c|}{Fix} &  30 & 12\\
%%                         &  \multicolumn{1}{c|}{$\neg$Fix} & 5 & 153 \\
%% \end{tabular}
%% \end{center}
%% \caption{\label{tbl-postgresql-confusion}PostgreSQL confusion matrix}
%% \end{table}





To validate our results, we estimated the precision and recall of our technique
for identifying bug-fixing commits on both projects. As our algorithm for
identifying the associated bug-introducing commits was a straightforward
application of git blame, we did not systematically verify its performance. (A
brief manual inspection of bug-introducing commits did not reveal any
anomalies.) For both projects, we randomly sampled 200 commits and manually
verified the results. Table~\ref{tbl-linux-confusion} and
\ref{tbl-postgresql-confusion} summarize our findings.

\begin{table}[tbh]
\centering
\small
\subfigure[{Linux kernel}]{
 \label{tbl-linux-confusion}
 \begin{tabular}{rrrr}
   & & \multicolumn{2}{c}{Predicted} \\
   & & \multicolumn{1}{|c}{Fix} & \multicolumn{1}{c}{$\neg$Fix} \\ \cline{2-4}
   \multirow{2}{*}{Actual} & \multicolumn{1}{c|}{Fix} & 48 & 18\\
                          &  \multicolumn{1}{c|}{$\neg$Fix} & 7 & 127 \\ \\
  \end{tabular}
}
\subfigure[PostgreSQL]{
  \label{tbl-postgresql-confusion}
  \begin{tabular}{rrrr}
    & & \multicolumn{2}{c}{Predicted} \\
    & & \multicolumn{1}{|c}{Fix} & \multicolumn{1}{c}{$\neg$Fix} \\ \cline{2-4}
    \multirow{2}{*}{Actual} & \multicolumn{1}{c|}{Fix} &  30 & 12\\
                            &  \multicolumn{1}{c|}{$\neg$Fix} & 5 & 153 \\ \\
  \end{tabular}
}
\caption{\label{tbl-confusion}Confusion matrices}
\end{table}


We evaluated the precision---that is, the proportion of identified bug-fixing
commits which do indeed fix bugs---and found that, for the Linux kernel, 48 of
the 55 bugs that we automatically identified as bug-fixing commits did indeed
fix bugs, while 7 did not; for PostgreSQL, 30 of 35 identified fixes were indeed
fixes. Some misclassifications included: 1) a commit message which fixed a merge
commit was classified as a fix; 2) apparently garbled commit messages which
included the keyword ``fix'' for no good reason; 3) changes which were reverted
(in the alleged ``fix'') but then re-added in a later version; 4) poor uses of
version control systems which included many different changes in a single
commit, including a fix as a small part of the commit; and 5) refactoring
changes, which moved or renamed functions; these could arguably be considered
to be fixes to a buggy initial design.

Our recall---the proportion of bug-fixing commits in the entire sample that our
technique identifies---is \linuxR for Linux and \postR for
PostgreSQL.

%\input{discussion}
%\section{Related Work}
\label{sec-related}

\paragraph{Day of Week of Commits}
The most closely related work studied the day of week of commits and
showed that the commits on Fridays are buggier~\cite{sliwerski-msr-2005}. 
This paper differs from it mainly in the following three aspects.
Firstly, we investigated how the commits' time of day correlates with the bugginess of commits, 
which has not been studied before to the best of our knowledge.
Secondly, we studied the correlation between commit bugginess and developers' activity levels as well as 
developer experience, which the previous work did not study.
Lastly, we used different data collection techniques. Specifically, we did not reply on the link 
between a commit and a bug report to extract bug-fixing commits, which enabled us to study 
software for which such links are not maintained or not well maintained by the developers. For example, we 
found that only 2.3\% of the bug-fixing commits are linked to a bug report, 
through manually examining a random sample of our bug-fixing commits in the Linux kernel.
While using links between bug reports and bug commits may increase the precision of extracting 
bug-fixing commits, our results demonstrate that 
high precision can be obtained without using such links: 
the precision of our bug-fixing commit extraction techniques are 
\linuxP for the Linux kernel and \postP for PostgreSQL.
%(1/43= 2.3\%)

\paragraph{Empirical Studies on Commits}
Previous study~\cite{largeCommits} classifies commits into different categories, one of which 
is non-functional commits (e.g., modification of comments, documentations, etc.).
Our study is different because we focused on
comment-fixing commits. For example, a refactoring commit 
%that modifies the source code but not comments 
would be considered as a non-functional commit by previous study, but not considered as a comment-fixing commits in this paper, 
thus being excluded from our study. 
iComment~\cite{iComment} only showed that FreeBSD contains many comment-fixing commits, and it is 
not a comprehensive study on comment-fixing commits.  
Our results show that XXX of commits in the Linux kernel and PostgreSQL only fix comments, and do not modify the code, showing
that developers spent time maintaining the correctness of comments. 

% cite ahmed hassan's work

\comment{ 
\paragraph{When do changed induce fixes?}

One of the relevant academic works is this paper, which analyzes CVS
achieves for fix-inducing changes. In other words, they examine code
changes that lead to problems. They discuss a methodology to
automatically locate fix-inducing changes by linking a version archive
to a bug database such as Bugzilla~\cite{sliwerski-msr-2005}. The authors
examined the history for Mozilla and Eclipse for their data
collection. The results they collected yielded that fix-inducing
changes show distinct patterns with respect to their size and the day
of the week they occurred. They discuss the general idea behind the
process of finding fix-inducing changes as: 1. Start with a bug report
in the bug database, indicating a fixed problem, extract the
associated change from the version archive to get the location of the
fix, and finally determine the earlier change at this location that
was applied before the bug was reported. The authors describe
syntactic and semantic analysis for the first step in the their
process of identifying potential fixes. Finally, through the use of
diff and annotate commands, as well as cycling through different
versions of the code, the authors are able to locate fix-inducing
changes. The study concluded that most bugs are introduced on Fridays
and Sundays. This work was important because it provided us with a
guideline for a methodology for extracting fix-inducing changes.

\paragraph{Automatic identification of bug-introducing changes}
The authors try to automate the process of finding bug-introducing
changes, which would remove the manual work associated with going
through bug reports or commit logs to collect this type of
information. They use the SZZ algorithm, which traces from the
location of the fix where the bug was introduced, and as such extract
the time \cite{2006-automatic}.The weakness with this algorithm is
that sometimes it makes false detections because not all modifications
are fixes, and a moderate improvement from using just SZZ is the use
of annotation graphs. Another improvement was ignoring format changes,
which reduce a large number of false positives. This paper has
inspired the approach we used for automating the process of
identifying bug-introducing changes, which was key for collecting a
large pool of data.

\paragraph{How long did it take to fix bugs?}
In this paper the author try to measure software quality as a function
of the number of bugs. The authors examine the bug fix time of files
in two open source software: ArgoUML and PostgreSQL, and tackle this
issue by identifying when bugs are introduced and when the bugs are
fixed \cite{2006-long}. The argument is files with the greatest
bug-fix times, whose bug counts are greater than average, may need
more attention to determine why bug fixes take such a long time –
potentially indicating the need for code refactoring to achieve faster
bug fixes in the future. The authors first extracted change histories
of the two projects, ArgoUML and PostgreSQL, using the Kenyon
infrastructure. To identify a bug-fix, they searched for keywords such
as fixed or bugs and they also searched for references to bug
reports. They then applied the identified bug-introducing changes by
applying the fix-inducing change identification algorithms described
in the paper “When do changes induce fixes?” which I mentioned
earlier.  This work is of course relevant because it gives us a
methodology for extracting bug-fix times.

\paragraph{If your bug database could talk}
This is another relevant work, where the authors perform experiments
that demonstrate how to relate developer, code, and process to defects
in the code. This work tries to understand why some programs are more
failure-prone than others.  To answer this question, we have to know
which programs are more failure-prone than others – to search for
properties of the programs or its development process that commonly
correlate with defect density. The authors try to answer questions
like “can one predict failure-proneness from metrics like code
complexity?”, “what does a high number of bugs found after release?”,
and “do some developers write more failure-prone code than
others?”. After examining Eclipse database, some of the conclusions
that the authors made were: new or combination of existing metrics
need to be explored to study the relationship between complexity of
code to the presence of bugs in a given class, it is difficult to
predict post-release failures solely from process measurements, and
there is a high variance in failure density in files owned by
different developers \cite{2006-if}. Their methodology for extracting
information from bug reports from BUGZILLA will be very useful for our
project.

\paragraph{On the Nature of Commits}
This paper studies the nature of commits in two dimensions: define the
size of commits in terms of number of files, and classify commits
based on the content of their comments \cite{hattori2008nature}. The
authors investigated the distribution of commits according to the
number of files, and their results show that the majority of commits
contain a large number of files. The authors also developed a
classification system for commits according to development and
maintenance activities based on the content of their commits, a system
that is more suitable for open source projects. Some of the major
findings made by the authors include: the majority of the commits are
not related to the development activities, corrective actions generate
more tiny commits, and development activities are spread among all
sizes of commits.
}

%\section{Conclusions and Future Work}
\label{sec-conclusion}

This paper analyzes \linuxBFC and \postBFC bug-fixing commits in two large and
widely-used open-source software projects, the Linux kernel and PostgreSQL
respectively, to study the correlation between commit correctness with several
commit social characteristics, such as the time-of-day of commits, the
day-of-week of commits, developer experience, and developers' activity
frequency. We found several interesting findings, including: (1) late-night
commits (between midnight and 4:00 AM) are buggier than average, while morning
commits (7:00 AM-noon) are less buggy, suggesting that developers may want to
double check late-night commits before committing, and that it may be beneficial
for the version control system to warn the developers of late-night commits to
improve software reliability; (2) our results show that the bugginess of commits
per day-of-week varies for different software projects, implying that the
bugginess prediction based on the day-of-week of commit metric may need to vary
on a project-by-project basis; and (3) developers who commit to a project on a
daily basis write less buggy commits for that project, while day-job developers
are more likely to produce bugs, indicating that we may want to promote the
practice of daily committing developers code-reviewing other developers'
commits.  We believe such results are valuable to the software engineering
community and software developers.

In the future, we would like to study commit times with respect to individual
developers to understand, for example whether a developer's commits outside of
his/her normal committing hours are buggier than average for that developer.  To
extend our developer experience study, we can add developers' contributions to
other open-source software projects to better understand a developer's overall
programming experience.  In addition, we plan to study more software projects
written in different programming languages to further understand how social
characteristics affect commit correctness.  As there may be interesting
correlations between a commit's evolution and its code quality, we intend to
study such correlations in the future.

%% \section{Future Work}
%% \label{sec-future}

%% There are a few parts of the implementation which could be improved to reduce
%% the number of false positives. First, our fix detection for commit messages is
%% very simplistic and could be improved to determine fixes which do not refer to
%% any bugs. We can also introduce additional logic to ignore code which has been
%% moved or renamed. The author information for experience also did not seem
%% reliable, due to the fact that we treat every name and e-mail pair as a unique
%% author. However, this might not be the case if the authors simply changed their
%% e-mail address.

%% We could also add additional extensions to the database including: categorizing
%% authors by their official roles, and classifying the size of each commit as well
%% to determine if larger commits are more likely to contain bugs.

%% In addition we would like to add additional software projects to the
%% database. Another goal is to release the data to the community so others may
%% extend and add to it. We believe there are much more interesting results which
%% could be found using our database.

%% \section{Conclusion}
%% \label{sec-conclusion}

%% Resolving bugs represent a substantial amount of time and cost in software
%% projects. It is important to investigate the cause of bugs in order to reduce
%% the amount of bugs which occur. We believe that the data we have collected will
%% be beneficial for the software quality assurance and software developing
%% personnel to use the correlations we found, and investigate other possible
%% correlations from the data we collected and stored in a database. From manually
%% checking a random sample we found our data to be representative. Our data has
%% revealed that Thursdays have a high bug introduction rate relative to the total
%% number of commits for that day, while Mondays have the lowest bug introduction
%% rates. There may be many reasons behind this correlation, but we believe it
%% might be that people are being worn out towards the end of the week, and as such
%% are more likely to introduce bugs, except this would mean that Fridays would be
%% worst - which is not the case. It could also be that on Fridays people are more
%% alert and focused because they are excited for the weekend. It was strange that
%% Mondays yielded the best results for time to code, but this could be the case
%% because people are just back from the weekend and they are fully charged, ready
%% to work. Another possible theory for Monday's results, which is that people
%% delegate easier tasks for Mondays because its the start of the week, or spend
%% more time planning and laying out templates of what they will be coding, and as
%% such are not as prone to introducing bugs. This of course could be proven by
%% checking the size of the commits and what kind of changes are being made on
%% Mondays. Our data also showed that coding between 12AM and 9AM is a bad idea,
%% while the best time is between 11AM and 3PM. This is not a very surprising
%% result, considering how most users sleep during these hours, and if they happen
%% to be doing any coding during these hours, it would be considered to be outside
%% of their usual working hours - making them more prone to errors.  From our
%% commits that were organized by author classification, it seems that daily
%% committers are less prone to bug introduction than those who are classified as
%% day job users. One theory that could potentially explain this result is that
%% hobbyist working on open source projects do it because they are highly
%% interested, while day job users are doing it to get payed, so the difference in
%% motivations could negatively affect the performance of the developer. Finally
%% our data shows that bug lifetimes decay exponentially, and on average are longer
%% for larger projects. This result of course is positive because it indicates that
%% the majority of bugs are dealt with fairly quickly.





{\small 
\bibliographystyle{abbrv}
\bibliography{paper}
}


%\theendnotes

\end{document}
