\section{Introduction}


Software demands high reliability. However, as its complexity increases, 
modern software inevitably contain bugs, which hurt software reliability.
The modern software development paradigm further complicates the situation: 
Many modern software, e.g., Firefox, the Linux kernel, Apache, and Eclipse, 
are developed by hundreds of or thousands of developers over decades~\cite{ccomment} 
in a distributed manner. The software often receives hundreds of commits per day (Section~\ref{sec-results}). 
Developers with different programming experiences, 
different time commitments, different programming styles, from all over the world,
work on the same software project at different times. They join and leave a project 
from time to time over decades.

Therefore, code developed in the modern paradigm has different social characteristics 
that are beyond the code itself. For example, different code segments in the same
software project may written by developers with  
varies level of expertise, different time 
commitment, different working hours, diverse culture, etc.
The code may be checked in at different times in different time zones. 

Such social characteristics provide a rich and unique source of information for us to 
understand the software and software bugs. Just to name one example, it would be beneficial to know if  
commit time (time of day, day of week, etc.) have implications on the qualify of the commit. 
Are commits submitted after midnight more buggy than other commits? 
Such correlation results may be useful to predict what commits are more likely
to be buggy so that we can budget more testing effort on these commits, just
like prior studies~\cite{graves00predicting, guo04robust, ostrand05predicting},
which predict bugs based on code complexity, 
code locations, amount of in-house testing, etc. 
Another interesting question is, whether more experienced developers are more or less
likely to introduce buggy commits. 


\subsection{Challenges}
There are several challenges in studying the correlation of commit correctness and software social characteristics.
First, it is challenging to know if a commit is buggy or not,
as bugs in the commit are often not 
discovered or reported until days, months, or even years after the commit. 
We use a reverse chronological method~\cite{sliwerski-msr-2005} to address this challenge. 
Specifically, we start from {\em bug-fixing commits}, commits that fix software bugs, 
and then leverage the version control system to understand when the corresponding bugs were introducted. 
%In other words, we extract the bug-introducing time for bug fixes 
%from the software version control systems. 
Then we can know collectively when bugs are more likely to be introduced. 
%In particular, we leverage the ``git blame'' functionality to identify the last modification time of each buggy line.

It is difficult to find bug-fixing commits in the sea of software commits.
The prior work~\cite{sliwerski-msr-2005} considers a commit whose commit message contains a bug 
ID that links to a bug report in a bug database to be a bug-fixing commit. While this
approach works for some software such as Mozilla, it does not work for software that 
rarely contain links from commit messages to bug reports, e.g., the Linux kernel.
We address this problem by applying heuristics to extract bug-fixing commits by examining
the commits themselves without analyzing bug reports at all. As shown later in Section~\ref{sec-results}, 
our heuristics have an accuracy of XXX\% in identifying bug-fixing commits.

Additionally, it is difficult to know how experienced the developers are in an open source development
environment as such information is not documented and subjective. 
We make our best-effort approximation by calculating the length and frequency of a developer
contributing to an open source project, which allows us to discover important findings 
and implications regarding developers' experience level and code correctness. 

\subsection{Contributions}
In this paper, we study the social characteristics of modern software to understand 
the correlation between these social characteristics with the correctness of the software.  
Specifically, we study the latest version of the Linux kernel and PostgreSQL with 222,332 and 31,098 commits respectively.
%XXX-XXX millions lines of code.
%, developed over XXX-XXX years. 
we study the correlation between a commit's bugginess
and the time of day of the commit, the day of week of the commit, and
the experience and activity frequency of the commit authors.
%We study the correlation between the commit correctness and the time of day of the commit, the day of week of the commit, 
%the committing developers' experience and activity frequency respectively.  
%...
%In addition to the above social characteristics, we also 
%In addition, we study several other commit characteristics, such as
%the commit sizes, XXX, XXX, etc. 

Our major findings 
are summarized below ($\S$ denotes in which
section the finding and its {\em implications} are discussed). To the best of our
knowledge, we are the {\em first} to study the correlation between the commit time of day and
the commit correctness. 

\begin{list}{\labelitemi}{\topsep=0pt\parsep=0pt\leftmargin=9pt\itemindent=0pt}

\vspace{0.05in}
\item {\bf Finding 1 ($\S$\ref{sec:finding1}):} 
Commits that are checked into the softare repository around midnight (between XXX-XXX AM) 
are XXX\% more likely to be incorrect, while commits around XXX-XXX AM are more likely to be correct.
The result indicate that developers may want to double check the code they write  
before they commit between these hours. It may also to beneficial for the version control
system to warn the developers of late night commits to improve software reliability. 

\item {\bf Finding 2 ($\S$\ref{sec:finding2}):} 
In contrary to the prior finding that Friday commits are buggier~\cite{2005-changes}, 
our results on the Linux kernel and PostgreSQL show that 
the bugginess differences of commits that are checked in on different days of week 
are not statistically significant. This result shows that the day of week of commits
vary for different software projects, implying that the bugginess prediction based on this 
metric may need to be on a project-by-project basis.

\item {\bf Finding 3 ($\S$\ref{sec:finding3}):} 
Develoeprs who commit to the repository on a daily basis
write less buggy commits, indicating that we may want to have daily
commiting developers code-review other developers' commits.

\end{list}


%Developers spend a large amount of time
%diagnosing bugs and fixing bugs by applying patches over the buggy software versions. 
%As modern software are often very complex and developed by thousands of developers 
%distributedly over many years, many of the patches are not correct, e.g., not fixing
%the bug, or introducing new bugs, etc~\cite{}. A recent study shows that XXX\% of patches
%are buggy. 
