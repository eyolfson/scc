\section{Introduction}

The area of bug detection and resolution is currently a very active area of research
in the software related disciplines. The process of detecting and resolving software
bugs is an extremely costly process because of the manpower required to perform
quality assurance on the developed software and having developers perform the fixes.
It is estimated that 20\% - 30\% of the time spent on a software project is spent on
testing and integration \cite{2004-industry}. In addition to the cost of manpower,
the time required to complete these tasks can be costly on the release date of the
software. As such, for the majority of software companies, the process of quality
assurance is costly and time consuming. To put the cost into perspective, a study
conducted by the National Institute of Standard and Technology in 2002 reported that
software bugs cost the economy of the United States approximately \$59.5 billion
annually \cite{2002-economic}. The study also showed that approximately 64\% of
the costs of software bug fixes are incurred by the end users, making this issue
of extreme importance for the consumers as well \cite{2002-economic}.

In our paper we will show the data collected from two large projects, which
are complex and representative of most software projects, namely Linux and Firefox.
Our data includes: percentage of bugs introduced vs total percentage of commits for
every day of the week, percentage of bugs introduced every hour vs percentage of total commits
every hour, percentage of introduction vs percentage of total commits by author classification,
percentage of bugs introduced vs percentage of total commits grouped by author experience,
and bug lifetime.

Our paper does not focus on solving a problem, but instead tries to find correlations between
bug introductions and days of the week, hours of the day, commit introductions by author
experience, commit introductions by author classification, and bug life time. This information
can be used to try and reveal any potential correlations that could help leaders
in the software industry determine ways to minimize any potential effects
of working on a certain day, or during a certain hour. Collecting and analyzing
data with regards to bug introduction will surely be beneficial to reducing bug
introduction rates by examining some of the potential causes.

This paper will be organized in the following fashion: Section
\ref{sec-idea} will give an insight to our idea and how we approached
the problem, Section \ref{sec-impl} discusses our implementation for
automatically finding bug introduction points, Section
\ref{sec-results} presents the results we found, Section
\ref{sec-related} will discuss the related academic work which our
work compliments, Section \ref{sec-future} explores possible future
work, and finally Section \ref{sec-conclusion} concludes our work.

