\section{Experimental Methods}
\label{sec:method}
Our overall goal is to investigate the properties of ``buggy'', or
bug-introducing, commits. We define a \emph{bug-introducing} commit to
be any commit for which there exists a later \emph{bug-fixing} commit
that purports to fix the bug. A single bug-fixing commit may fix bugs
introduced in multiple bug-introducing commits. Despite our
terminology, a bug-introducing commit is not necessarily bad code; it
is possible that the later fix is adaptive or perfective, updating the
code to work with changes in third-party code, or reflecting a change
in requirements.

\subsection{Core Methods}
Following~\cite{sliwerski-msr-2005}, our methodology has three steps: 1) enumerating bug-fixing commits; 2)
identifying the lines changed in each bug-fixing commit; and 3)
finding the commits which were responsible for the previous (buggy)
version of each of the changed lines. We describe each of these steps
in more detail. First, to detect bug-fixing commits $c$, we searched
the commit messages for the keyword ``fix'' (as do~\cite{smallCommits05}). 
In our experience, most developers indicate that a change is a fix by including the keyword
``fix'' in the commit message; Section~\ref{sec-results} explains how
we verified this intuition, and the results show that the precision
of this heuristic for identifying bug-fixing commits is \postP-\linuxP. 
Next, we computed diffs for each file
changed in $c$ and recorded the line numbers $L_c$ which $c$
changed that are not comments. Finally, we searched the repository %% Jon: specified that we ignore comments here
metadata (as implemented by the ``git blame'' command) to identify the
bug-introducing commits $c'$ which changed the lines $L_c$. In this
step, we used git blame's {\tt -w} option, which ignores whitespace in
attributing responsibility for a code change, which addresses some
inaccuracies that exist in other version control systems such as CVS
and Subversion as discussed by Kim et. al~\cite{2006-automatic}.

\paragraph{Example} Listing~\ref{lst-fix} presents an example of a
bug-fixing commit. Commits consist of a commit id, which is a hash
of the commit's contents, represented as a hexadecimal number\footnote{Following common practice, because we can uniquely identify the commit by the first 8 digits of its commit id, we omit the other digits.}; an
author, identified by a name/email address pair; a commit message,
which contains the keyword ``fix''; and a diff, illustrating the lines
that the commit modified. 

First, we would find the commit in Listing~\ref{lst-fix} by searching
the commit logs for the keyword ``fix'', which is indeed a substring
of the commit message, ``I fixed a bug!''.  Next, we would observe
that the commit modifies line 100 of an (unidentified) file---in this
case, the original author used less-than-or-equal (\verb+<=+) instead
of strictly-less-than (\verb+<+), and the bug-fixing commit changes the
comparison operator to the presumably-correct one.  Finally, we
perform a ``git blame'' on this commit, which shows the commit
responsible for the previous version of line 100. Listing
\ref{lst-blame} presents plausible ``git blame'' output, which shows
that the source of the bug fixed in Listing~\ref{lst-fix} is the
commit whose id begins with f4ce718c, as shown in
Listing~\ref{lst-introduction}. We flag this commit as a
bug-introducing commit and store both the bug-introducing
commit f4ce718c and its bug-fixing commit 2cdc03fe in the database,
along with an association between these two commits.

\begin{lstlisting}[caption=An example bug-fixing commit,label=lst-fix,frame=single]
Commit: 2cdc03fe...
Author: Alice <alice@project.com>
Message: I fixed a bug!
@@ -100,1 +100,1 @@
-    if (i <= 128) {
+    if (i < 128) {
    \end{lstlisting}

    \begin{lstlisting}[caption={\tt git blame} output for the bug-fixing commit,label=lst-blame,frame=single]
f4ce718c...  100    if (i <= 128) {
    \end{lstlisting}

    \begin{lstlisting}[caption=Associated bug-introducing commit for the example,label=lst-introduction, frame=single]
Commit: f4ce718c...
Author: Bob <bob@project.com>
Message: I hope this works.
@@ -100,0 +100,5 @@
+    if (i <= 128) {
+        do_ascii(i);
+    else {
+        do_unicode(i);
+    }
    \end{lstlisting}

Any change in $c$ which removes or modifies an existing line
of code is easy to attribute to a previous commit $c'$, since the
affected line of code existed in $c'$. However, a change in $c$ which
adds a new line of code has no corresponding change in any previous
revision, since that line did not previously exist. In that case, we
attribute responsibility to the commit which introduced the line just
before the new line. This heuristic does not work for bug-introducing
changes in newly-introduced files. Fortunately, our data show that 
such changes are extremely rare, so we can ignore them entirely.

\subsection{Data Collected}
\label{sec:data}
Executing the above algorithm gives us data about the bug-fixing
and bug-introducing commits
in each repository, as well as about the authors of these commits.  
%\paragraph{Basic Meta-Information}
We record the following data for each commit: author (as a name/email
pair); adjusted time (as described below); number of lines changed; 
and number of times the
commit introduced a bug later corrected (which is derived data; we
record it to simplify later database queries). We also record a
relation connecting bug-introducing commits and bug-fixing
commits. For each author, we record the name, email(s), experience and
frequency classification. We define a bug lifetime's to be the time from the
earliest commit which introduced the bug to the bug-fixing commit.


%\paragraph{Author Experience and Activity Frequency}
We compute an author's activity frequency and experience classification based on the
frequency and duration of that author's commits to a particular
project. 

{\em Author activity} classification
describes the author's most-common frequency between consecutive
commits: daily, weekly, monthly, other (less than 20 commits), and
single. For author activity classification, we count consecutive
commits within 30 minutes of each other as one commit. As a sub-class
of the daily commiter classification, we also use a heuristic to
identify commiters who appear to work on the repository as part of
their day job, namely those for which 85\% of commits are between 8 AM
and 4 PM Monday to Friday.

{\em Author experience} counts the time between an author's
earliest and latest commits to the project. 
To study the correlation between the commit bugginess and 
the author experience, we calculate the author's experience at the time of the commit, 
and group commits based on their author experience. 
Then for each group of commits, we calculate the percentage of buggy commits. 
 

\paragraph{Time zone adjustments}
All Postgres commits prior to September 18, 2010 (when the project
switched to the Git version control system) contain timestamps in UTC
(Coordinated Universal Time). To enable us to reason about time-of-day
effects for committers, who work in local time zones, we used
publicly-available information (such as the Postgres
contributor-information page, which lists locations for frequent
contributors, as well as time zones included in mailing list messages) to deduce
time zones for all 34 Postgres committers. We then used the Python
time zone utilities to convert the time for each commit (which was in
UTC for CVS commits, and in local times for git commits) into a local
time for the committer. We assume that each committer is indeed in
the time zone we have estimated for that committer; 
we discuss this threat to validity in greater detail below.

\subsection{Threats to Validity}
We discuss a number of threats to validity and how we address them,
including general threats to construct validity and external validity,
and specific threats to our particular methodology, including 
repository threats, recall and precision threats, 
and author identification threats.

\paragraph{General threats}
A first threat to (construct) validity is that we must properly
identify bug-fixing and bug-introducing commits. To assess this
threat, we estimate the false positive rate by randomly sampling
commits that our analysis identified to be bug fixes and manually
verifying whether or not they are indeed fixes.  We also estimate the
false negative rate by randomly sampling commits identified not to be
bug fixes and manually verifying that these commits are not fixes.
Section~\ref{sec-results} presents our evaluation of precision and recall 
in greater detail.

While we believe that the commits from the software that we examined
well represent commits in open source software, we do not intend to
claim external validity and draw any general conclusions about all
software.  Like any other characteristic study, our findings should be
considered together with our evaluation methodology.

\paragraph{Repository data threats}
We expect our methodology to properly account for developers in
different time zones. Git records each developer's local time (and
time zone) with a commit, thereby avoiding potential imprecisions in
our time-of-day results. This works well for the Linux kernel repository,
which is a native Git repository as of 2005. (Older Linux repository
information does not accurately record time-of-day information for
commits.) In that repository, the
local time might be inaccurate if a developer commits from a server in
a different time zone; however, we expect that most of the 8000 Linux kernel
contributors will commit locally on a machine with accurate local
time.  The accuracy of the commit times for the Postgres repository
depends on the accuracy of our time adjustment algorithm for the part
of the history that was originally a CVS history. We believe that
committers don't often change time zones\footnote{We were able to identify
one move of a committer from Ontario to California, and incorporated that
move into our adjustments, but did not find evidence of many such moves
in our set of Postgres contributors.} and that they usually work from their
home time zone, but the validity of our adjustment does depend on the
validity our assumptions about home time zones. We present all of our
results in the local time of the committer.

Because the Postgres repository was converted from CVS to git, the
accuracy of the conversion is another potential threat to validity.  In
particular, CVS does not have a notion of atomic multi-file commits,
while Git does, and our methodology relies on the existence of such
commits. The Postgres conversion used the standard {\tt cvs2git} tool,
with customizations for their particular
repository~\cite{haas09:_so_why_postg_using_git}. We have reasons to
believe that the conversion was performed with care, therefore
mitigating threats to validity from repository corruption. 
Also, note that the conversion to git obviates the need to mine transactions
from CVS histories, as in~\cite{zimmermann-msr-2004}.

\paragraph{Threats due to imperfect recall}
Because our methodology only identifies bugs which have later been
corrected, it will definitely omit recently-introduced bugs, as well
as longer-running bugs which have not yet been corrected. Furthermore,
our recall of \postR for Postgres and \linuxR for the Linux kernel means
that our methodology does not manage to identify some of the bug fixes
in the repository histories.

While these threats imply that our results will omit some bugs, we do
not believe that this omission affects our validity. There is no
reason to believe that there are important differences in the
characteristics we measure for fixed and unfixed bugs, nor between
fixes labelled ``fix'' and fixes without that label. In other words,
we believe that our sample is representative of bug-introducing and
bug-fixing commits for these software projects.

\paragraph{Threats to commit characteristics}
The next set of related threats concerns commit characteristics,
including the attributed time and author information for a patch.
These threats do not apply to the Linux kernel repository, as the Linux
kernel community uses the merge functionality of Git to preserve
commit metadata from external contributors. (Postgres committers, by
contrast, manually merge diffs contributed by external contributors).
The implications of Linux's Git merges on our data are that 1) the commit
time reflects the initial author's commit decision, and 2) the
attributed author of a patch actually wrote the patch.

Most Postgres patches are committed by someone other than the author
of the patch: the set of Postgres committers is quite small, so
external Postgres commits will always be attributed to a member of the
Postgres team. Although Git can record both the author and committer
of a patch, the Postgres community is currently requiring that the
``author'' field of a patch always equal its ``committer'' field.
Third-party commits add more uncertainty about when a patch was
originally written, and confound our data about patch authors.  We
believe, however, that even for Postgres, it is reasonable to ascribe
responsibility for the patch to its commiter, who would presumably
review a patch before committing it. 

Developers work on a patch~\footnote{We use ``patch" and ``commit" interchangeably in this paper.} over a possibly-discontiguous time
interval. However, the final commit only indicates the endpoint of
that interval.  The bug database may contain more information about
the starting point of the interval (e.g. it records when a bug is
assigned to a developer), but still does not capture any information
about the work patterns of the developer within the interval. Some Git
repositories, including some Linux kernel sub-repositories, do contain more
information about intermediate local commits by developers, which can
help understand the evolution of a commit. Since there may be
correlations between a patch's evolution and its code quality, we
intend to investigate these issues in future work.

Finally, we discuss threats to our author classification. Some Linux kernel 
authors commit patches from several email addresses; we treat each
email address as a separate author, which may cause some authors to be
appear twice, in two separate classifications. If an author has the
same email but two different names, we merge the author information
for the two sets of commits. This is not an issue for Postgres: when
they converted their repository to git, they also normalized all author
email addresses, so that each author has a single email address.

Our day-job classification for authors is based on the fixed hours of
8AM-4PM and Monday-Friday days of the week.  Many software companies
have flexible hours; however, we expect that developers who are
directed to work on a project as their day job would still have the
bulk of their commits between those hours.  The day-job classification
applies only to the Linux kernel committers; no Postgres committers satisfy
the day-job criteria. Even though most of the core Postgres team have
Postgres-related employment, we believe that they are not directed to
work on Postgres, which would account for the observation that most of
their commits to Postgres occur outside normal working hours.

Our author experience information only counts the participation of an
author to a particular project; a highly-experienced developer may be
classified as an inexperienced developer for one of our projects due
to a short history of contributions to that project. We expect,
however, that there is some value to characterizing authors by their
participation in a single project when analyzing the quality of their
contributions to that project. It would be possible to survey
developers' participation across multiple projects by studying the
public record, but such a study is beyond the scope of this paper.
