\section{Experimental Methods}
\label{sec:method}
Our goal is to find commits which report fixing a bug, as well as the commits
which introduced that bug. First, we detect a bug-fixing commit $c$ by
searching the commit messages for common keywords such as ``fix'' and
``bug''. Next, we work backwards to locate the commits that introduced
the bug fixed in $c$. Using commit $c$ as a starting point, we compute
a diff for each file between the version after $c$ and the previous
version of that file, and record the line numbers changed in
$c$. Finally, we use the repository metadata to find the commit(s)
$c'$ which were responsible for the previous versions of the files
which were fixed, and we conclude that the commits $c'$ introduced the
bug.

Consider the commit in Listing \ref{lst-fix} (presented as a diff),
which we find using ``fix'' as a keyword, the line which caused the
bug is line 100 (in this case the original author used less than or
equal to instead of strictly less than). We found most developers
indicate that their change is a fix by including the keyword ``fix''
in the commit message. We perform a blame on this commit, which is
shown in Listing \ref{lst-blame}, and determine that its source is the
commit starting with f4ce718c (which is shown in Listing
\ref{lst-introduction}. We add this commit as a bug-introduction
commit for this bug-fix and store it in the database.

    \begin{lstlisting}[caption=An example bug-fix,label=lst-fix,frame=single]
Commit: 2cdc03fe...
Author: Alice <alice@project.com>
Message: I fixed a bug!
@@ -100,1 +100,1 @@
-    if (i <= 128) {
+    if (i < 128) {
    \end{lstlisting}

    \begin{lstlisting}[caption=Blame of the bug-fix,label=lst-blame,frame=single]
f4ce718c...  100    if (i <= 128) {
    \end{lstlisting}

    \begin{lstlisting}[caption=An example bug-introduction,label=lst-introduction, frame=single]
Commit: f4ce718c...
Author: Bob <bob@project.com>
Message: I hope this works.
@@ -100,0 +100,5 @@
+    if (i <= 128) {
+        do_ascii(i);
+    else {
+        do_unicode(i);
+    }
    \end{lstlisting}

Note that any change in $c$ which removes or modifies an existing line
of code is easy to attribute to a previous commit $c'$, since the
affected line of code existed in $c'$. However, a change in $c$ which
adds a new line of code has no corresponding change in any previous
revision, since this line did not previously exist. In that case, we
attribute responsibility to the commit which introduced the line just
before the new line.

\paragraph{Data collected}
Executing the above algorithm gives us data about a number of commits
in the repository, as well as about the authors of these commits.  We
record the following data for each commit: author (as a name/email
pair), time, number of lines changed, and number of times the commit
introduced a bug later corrected. For each author, we record the name,
email(s), experience and classification. We define a bug lifetime's to
be the time from the earliest commit which introduced a bug to the
bug-fixing commit.

We compute an author's experience and classification based on the
frequency and duration of that author's commits to a project. Author
experience counts the time between an author's earliest and latest
commits to the project. Author classification describes the author's
most-common frequency between consecutive commits: daily, weekly,
monthly, and single. Note that, for author classification purposes, we
ignore consecutive commits within 30 minutes of each other. As a
sub-class of the daily commiter classification, we also use a
heuristic to identify commiters who worked on the repository as part
of their day job, namely those for which 85\% of commits are between 8
AM and 4 PM Monday to Friday.

\paragraph{Threats to Validity}
We describe a number of threats to validity: ....

A first threat to (construct) validity is that we must properly
identify bug-fixing and bug-introducing commits. To assess this
threat, we randomly sample commits that our analysis identified to be
bug fixes and manually verify whether or not they are fixes to
determine the false positive rate. Vice-versa, we randomly sample
commits identified not to be bug fixes and manually verify these
results to determine the false negative rate.

While we believe that the commits from the examined software well
represent commits in open source software, we do not intend to draw
any general conclusions about all software.  Like any other
characteristic study, our findings should be considered together with
our evaluation methodology.

We expect our methodology to properly account for developers in
different time zones. Git records each developer's local time (and
time zone) with a commit, thereby avoiding potential imprecisions in
our time-of-day results. We present all of our results in the local
time of the committer.

Because our methodology only identifies bugs which have later been
corrected, it will definitely omit recently-introduced bugs, as well
as longer-running bugs which have not yet been corrected. While this
implies that our results will omit some bugs, we do not believe that
this omission affects our validity, because there is no reason to
believe that there are important differences in the characteristics we
measure for fixed and unfixed bugs.

Finally, developers work on a patch over a (possibly-discontiguous)
time interval, but the code repository only records the endpoint of
that interval. The bug database may contain more information about the
starting point of the interval (e.g. it records when a bug is assigned
to a developer), but still does not capture any information about the
work patterns of the developer within the interval. There may be
correlations between the coding time (the width of the interval) and
code quality. We intend to investigate these issues in future work.
