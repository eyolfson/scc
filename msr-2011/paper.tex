% This is "sig-alternate.tex" V1.9 April 2009
% This file should be compiled with V2.4 of "sig-alternate.cls" April 2009
%
% ----------------------------------------------------------------------------------------------------------------
% This .tex file (and associated .cls V2.4) produces:
%       1) The Permission Statement
%       2) The Conference (location) Info information
%       3) The Copyright Line with ACM data
%       4) NO page numbers
%
% as against the acm_proc_article-sp.cls file which
% DOES NOT produce 1) thru' 3) above.
%
% Using 'sig-alternate.cls' you have control, however, from within
% the source .tex file, over both the CopyrightYear
% (defaulted to 200X) and the ACM Copyright Data
% (defaulted to X-XXXXX-XX-X/XX/XX).
% e.g.
% \CopyrightYear{2007} will cause 2007 to appear in the copyright line.
% \crdata{0-12345-67-8/90/12} will cause 0-12345-67-8/90/12 to appear in the copyright line.
%
%
% For tracking purposes - this is V1.9 - April 2009

\documentclass{sig-alternate}

\pdfpagewidth 8.5in
\pdfpageheight 11in

\clubpenalty=100000000
\widowpenalty=100000000
%\brokenpenalty=100000000

\usepackage{times}
\usepackage{url}

%% Define a new 'leo' style for the package that will use a smaller font.
\makeatletter
\def\url@leostyle{%
  \@ifundefined{selectfont}{\def\UrlFont{\sf}}{\def\UrlFont{\small\ttfamily}}}
\makeatother
%% Now actually use the newly defined style.
\urlstyle{leo}

%------------------------------------------------------------------------- 
% take the % away on next line to produce the final camera-ready version 
\pagestyle{empty}

%------------------------------------------------------------------------- 
\usepackage{multirow}
\usepackage{arydshln}
\usepackage{boxedminipage}
\usepackage{xspace}
\usepackage{listings}
\usepackage[TABBOTCAP]{subfigure}

%%Squeeze space in bibliographybg
%  \let\oldthebibliography=\thebibliography
%  \let\endoldthebibliography=\endthebibliography
%  \renewenvironment{thebibliography}[1]{%
%    \begin{oldthebibliography}{#1}%
%      \setlength{\parskip}{0ex}%
%      \setlength{\itemsep}{0ex}%
%  }%
%  {%
%    \end{oldthebibliography}%
%  }

\lstset{language=C}

\newcommand{\code}{\tt \small}

\newcommand{\comment}[1]{{}}
\newcommand{\note}[1]{{\bf *** NOTE: {#1} ***}}
%\newcommand{\bfc}{bug-fixing commits}
%\newcommand{\bic}{bug-introduction commits}
\newcommand{\linuxP}{87\%\xspace} %precision for linux
\newcommand{\linuxR}{73\%\xspace} % recall for linux, lower bound
\newcommand{\postP}{86\%\xspace} % precision ofr postgreSQL
\newcommand{\postR}{71\%\xspace} % recall for postgreSQL, lower bound
\newcommand{\linuxBFC}{57,028\xspace} %total number of bug-fixing commits for linux
\newcommand{\postBFC}{4,399\xspace} % total number of bug-fixing comits for postgreSQL 
\newcommand{\fbuggy}{1\xspace} % finding on day of week
\newcommand{\fhour}{2\xspace} % finding on day of week
\newcommand{\fdaily}{3\xspace} % finding on day of week
\newcommand{\fday}{4\xspace} % finding on day of week

\def\sharedaffiliation{%
\end{tabular}
\begin{tabular}{c}}

\begin{document}
%
% --- Author Metadata here ---
\conferenceinfo{MSR}{'11, May 21-22, Waikiki, Honolulu, Hawaii, USA}
\CopyrightYear{2011}
\crdata{978-1-4503-0574-7/11/05}
% --- End of Author Metadata ---

\title{Do Time of Day and Developer Experience Affect Commit Bugginess?}
%% \subtitle{Subtitle Text, if any}


%\title{Alternate {\ttlit ACM} SIG Proceedings Paper in LaTeX
%Format\titlenote{(Produces the permission block, and
%copyright information). For use with
%SIG-ALTERNATE.CLS. Supported by ACM.}}
%\subtitle{[Extended Abstract]
%\titlenote{A full version of this paper is available as
%%\textit{Author's Guide to Preparing ACM SIG Proceedings Using
%\LaTeX$2_\epsilon$\ and BibTeX} at
%\texttt{www.acm.org/eaddress.htm}}}
%
% You need the command \numberofauthors to handle the 'placement
% and alignment' of the authors beneath the title.
%
% For aesthetic reasons, we recommend 'three authors at a time'
% i.e. three 'name/affiliation blocks' be placed beneath the title.
%
% NOTE: You are NOT restricted in how many 'rows' of
% "name/affiliations" may appear. We just ask that you restrict
% the number of 'columns' to three.
%
% Because of the available 'opening page real-estate'
% we ask you to refrain from putting more than six authors
% (two rows with three columns) beneath the article title.
% More than six makes the first-page appear very cluttered indeed.
%
% Use the \alignauthor commands to handle the names
% and affiliations for an 'aesthetic maximum' of six authors.
% Add names, affiliations, addresses for
% the seventh etc. author(s) as the argument for the
% \additionalauthors command.
% These 'additional authors' will be output/set for you
% without further effort on your part as the last section in
% the body of your article BEFORE References or any Appendices.

%% \numberofauthors{1} %  in this sample file, there are a *total*
%% % of EIGHT authors. SIX appear on the 'first-page' (for formatting
%% % reasons) and the remaining two appear in the \additionalauthors section.
%% %
%% \author{
%% % You can go ahead and credit any number of authors here,
%% % e.g. one 'row of three' or two rows (consisting of one row of three
%% % and a second row of one, two or three).
%% %
%% % The command \alignauthor (no curly braces needed) should
%% % precede each author name, affiliation/snail-mail address and
%% % e-mail address. Additionally, tag each line of
%% % affiliation/address with \affaddr, and tag the
%% % e-mail address with \email.
%% %
%% % 1st. author
%% \alignauthor
%% Jon Eyolfson, Lin Tan, Patrick Lam\\ 
%%        \affaddr{University of Waterloo} \\
%%        \affaddr{200 University Avenue West}\\
%%        \affaddr{Waterloo, ON N2L3G1, Canada} \\ 
%%        \affaddr{\em jon@eyolfson.com, lintan@uwaterloo.ca, p.lam@ece.uwaterloo.ca}
%% }


\numberofauthors{3}
\author{
  \alignauthor Jon Eyolfson\\
  \email{jeyolfso@uwaterloo.ca}
  \alignauthor Lin Tan\\
  \email{lintan@uwaterloo.ca}
  \alignauthor Patrick Lam\\   
  \email{p.lam@ece.uwaterloo.ca}
  \sharedaffiliation
  \affaddr{University of Waterloo}\\
  \affaddr{200 University Avenue West}\\
  \affaddr{Waterloo, Ontario, Canada N2L3G1}
}

\maketitle

\begin{abstract}
%% Notes:
%% - should we clarify what daily means in this context?

Modern software is often developed over many years with hundreds of
thousands of commits. Commit metadata is a rich source of social
characteristics, including the commit's time of day and the
experience and commit frequency of its author.  The ``bugginess'' of a
commit is also a critical property of that commit. In this paper, we
investigate the correlation between a commit's social characteristics
and its ``bugginess''; such results can be very useful for software
developers and software engineering researchers. For instance,
developers or code reviewers might be well-advised to thoroughly
verify commits that are more likely to be buggy.

In this paper, we study the correlation between a commit's bugginess
and the time of day of the commit, the day of week of the commit, and
the experience and commit frequency of the commit authors.  We
survey two widely-used open source projects: the Linux kernel and
PostgreSQL.  

Our main findings include: (1) commits submitted between midnight and
4 AM (referred to as late-night commits) are significantly buggier 
and commits between 7AM and noon are less buggy, implying that developers 
may want to double-check their own late-night commits; 
(2) daily committing developers produce less-buggy commits, indicating that we may 
want to promote the practice of daily committing developers code-reviewing other 
developers' commits; and (3) 
the day of week of commits
vary for different software projects, implying that the bugginess prediction based on 
the day of week of commits may need to be on a project-by-project basis.

\end{abstract}

\category{D.2.7}{Software Engineering}{Distribution, Maintenance, and Enhancement}
\category{D.2.9}{Software Engineering}{Management}

\terms
Human Factors, Management, Measurement

\keywords
Bug Detection, Empirical Study, Source Control System

\maketitle

% There's nothing stopping you putting the seventh, eighth, etc.
% author on the opening page (as the 'third row') but we ask,
% for aesthetic reasons that you place these 'additional authors'
% in the \additional authors block, viz.
% \additionalauthors{}
%\date{30 July 1999}

% Just remember to make sure that the TOTAL number of authors
% is the number that will appear on the first page PLUS the
% number that will appear in the \additionalauthors section.


% A category with the (minimum) three required fields
%\category{}{}{}[]
%\category{}{}{}[]

%\category{H.4}{Information Systems Applications}{Miscellaneous}
%A category including the fourth, optional field follows...
%\category{D.2.8}{Software Engineering}{Metrics}[complexity measures, performance measures]

%\terms{Theory}

%\keywords{Empirical Study, Bug detection}


%------------------------------------------------------------------------------

\input intro
\input methods
%\input idea % idea, design, and methods
\input results
\input related
\input conclusions

{\small 
\bibliographystyle{abbrv}
\bibliography{paper}
}



%\note{fix bad breaks}

%\note{Check against reviewers' comments}

%\theendnotes

\end{document}
