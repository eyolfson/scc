 
%% Notes:
%% - I removed the commented out cite~\cite{ccomment}
%\note{cite~\cite{updateComment}} not that related

\section{Introduction}

\note{better explanation of 8-4, clarify our definition of day-job}

\note{wait for 2 hours to commit}

\note{fix bad breaks}

\note{Check against reviewers' comments}

\hyphenation{Apa-che}
Software users demand high software reliability. However, as software complexity
increases, bug counts and rates inevitably rise, which undermine software
reliability. The modern software development paradigm further complicates the
situation: many modern software projects, including the Linux kernel,
PostgreSQL, Eclipse, and Apache, are developed by tens to thousands of
developers, over decades, in a distributed manner. The software often receives
tens of thousands or hundreds of thousands of commits
(Section~\ref{sec-results}). Developers with different programming experience,
different time commitments, different working hours, different programming
styles, and from diverse cultures from all over the world, work on the same
software project at different times and in different time zones. They join and
leave projects at their own pace over periods of decades. Code developed in the
modern paradigm can therefore have different social characteristics from older,
more homogeneously-developed projects; these characteristics can best be
measured by going beyond the code itself and into the social characteristics of
the code.
% and into code repositories.

Software social characteristics provide a rich and unique source of information
for us to understand software and its bugs. As an example, it would be helpful
to know if a commit's timestamp (including features such as time of day, day of
week, etc.)  affects the quality of that commit --- are commits submitted after
midnight buggier than other commits?  Such correlations may be useful for
predicting what commits are more likely to be buggy so that we can budget more
testing effort on these commits, following prior
studies~\cite{graves00predicting, guo04robust, Hassan09, libre07, devNetwork08,
  predictionMenzies10, effort03, ostrand05predicting, depGraph08,
  zimmermann-promise-2007}, which predict buggy locations based on code
complexity, code locations, the amount of in-house testing, historical data,
socio-technical networks, etc. A second interesting question is whether more
experienced developers are more or less likely to write buggy commits.

%% Additionally, it is difficult to know how experienced the developers are in an
%% open source development environment as such information is often not documented
%% and subjective.  We make our best-effort approximation by calculating the
%% duration of a developer contributing to an open source project (i.e., the time
%% between a developer's earliest and latest commits to the project), which allows
%% us to discover important findings and implications regarding developers'
%% experience level and code correctness.

\paragraph{Contributions}

In this paper, we study the social characteristics of modern software
development to understand the correlation between these social characteristics
and the bugginess of commits to the software---the likelihood that a
particular commit is later fixed, as determined by the fixing author.
Specifically, we study the latest versions of the Linux kernel and PostgreSQL,
which have 222,332 and 31,098 commits, respectively. We study the correlation
between a commit's bugginess and the time of day of the commit, the day of week
of the commit, and the experience and activity frequency of the commit authors.
In addition, we study several other commit characteristics, such as comment-only
fixes and bug lifetimes.  To the best of our knowledge, we are the {\bf first}
to study the correlation between the commit time of day and the commit
correctness.

To study the correlation between commit time and commit bugginess, we start from
{\em bug-fixing commits}, commits that fix software bugs, and then leverage the
version control system to discover when the corresponding bugs were
introduced~\cite{sliwerski-msr-2005}.  Our methodology enables us to observe
when bugs are more likely to be introduced.  Note that we simply use ``bug'' to
denote code that is later changed, even though such code may objectively be
correct; we expand on this discussion later, in Section~\ref{sec:method}.

It is difficult to find bug-fixing commits in the sea of software commits.
Prior work~\cite{sliwerski-msr-2005} defines a bug-fixing commit to be a commit
whose commit message contains a bug ID that links to a bug report in a bug
database. While this approach works for some projects, like Mozilla, it does not
work for software whose commit messages rarely contain links to bug reports,
like the Linux kernel.  We have observed that only 2.3\% of the bug-fixing
commits in the Linux kernel are linked to a bug report. We address this problem
by applying heuristics that scan commit messages; they do not rely on any links
between bug commits and bug reports to extract bug-fixing commits.  Our
heuristics have a precision of \postP-\linuxP in identifying bug-fixing commits
(Section~\ref{sec-results}).

Our major findings are summarized below ($\S$ denotes the section where the
finding and its {\em implications} are discussed):

\begin{list}{\labelitemi}{\topsep=0pt\parsep=0pt\leftmargin=9pt\itemindent=0pt}

\item {\bf Finding \fbuggy ($\S$\ref{sec-proj-char}):} 
About a quarter (23.7-25.5\%) of all the commits in the Linux kernel and
PostgreSQL are labelled buggy---they require further developer activities to fix
them.

\item {\bf Finding \fhour ($\S$\ref{sec-time-of-day}):} 
Commits that are checked into the software repository around midnight (between
0:00-4:00 AM) are more likely to be incorrect than average, while commits in the
morning (7:00 AM-noon) are more likely to be correct.  The result indicate that
developers may want to double check the code they write for these late-night
commits (0:00-4:00 AM).  It may also be beneficial for the version control
system to warn the developers of late-night commits to improve software
reliability.

\item {\bf Finding \fdaily ($\S$\ref{sec-dev-char}):} 
Developers who commit to the repository on a daily basis write less-buggy
commits, while developers who appear to work on a project as part of their
day-job are more likely to produce bugs, indicating that we may want to promote
the practice of daily-committing developers code-reviewing other developers'
commits.

\item {\bf Finding \fday ($\S$\ref{sec-day-of-week}):} 
In contrast to a prior finding that Friday commits are
buggier~\cite{sliwerski-msr-2005}, our results on the Linux kernel and
PostgreSQL show that the bugginess differences of commits that are checked in on
different days of week are slight. Our result shows that the bugginess per
day-of-week for commits varies for different software projects, implying that
bugginess prediction based on this metric may need to be on a project-by-project
basis.

\end{list}
