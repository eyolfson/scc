\section{Conclusions and Future Work}
\label{sec-conclusion}

This paper analyzed \linuxBFC and \postBFC bug-fixing commits in two large and
widely-used open-source software projects, the Linux kernel and PostgreSQL, 
to study the correlation between commit correctness with several
commit social characteristics, such as the time-of-day of commits, the
day-of-week of commits, developer experience, and developers' commit 
frequency. We presented several interesting findings, including: (1) late-night
commits (between midnight and 4:00 AM) are buggier than average, while morning
commits (7:00 AM--noon) are less buggy, suggesting that developers may want to
double-check late-night commits before committing, and that it may be beneficial
for the version control system to warn the developers of late-night commits to
improve software reliability; (2) the bugginess of commits
per day-of-week varies for different software projects, implying that the
bugginess prediction based on the day-of-week of commit metric may need to vary
on a project-by-project basis; and (3) developers who commit to a project on a
daily basis write fewer buggy commits for that project, while day-job developers
are more likely to produce bugs, indicating that we may want to promote the
practice of daily committing developers code-reviewing other developers'
commits. We believe such results are valuable to the software engineering
community and software developers.

In the future, we would like to study commit times with respect to individual
developers to understand, for example, whether a developer's commits outside of
his/her normal committing hours are buggier than average for that developer. To
extend our developer experience study, we can add developers' contributions to
other open-source projects to better understand a developer's overall
programming experience. In addition, we plan to study more software projects
written in different programming languages to further understand how social
characteristics affect commit correctness. As there may be interesting
correlations between a commit's evolution and its code quality, we intend to
study such correlations in the future.

%% \section{Future Work}
%% \label{sec-future}

%% There are a few parts of the implementation which could be improved to reduce
%% the number of false positives. First, our fix detection for commit messages is
%% very simplistic and could be improved to determine fixes which do not refer to
%% any bugs. We can also introduce additional logic to ignore code which has been
%% moved or renamed. The author information for experience also did not seem
%% reliable, due to the fact that we treat every name and e-mail pair as a unique
%% author. However, this might not be the case if the authors simply changed their
%% e-mail address.

%% We could also add additional extensions to the database including: categorizing
%% authors by their official roles, and classifying the size of each commit as well
%% to determine if larger commits are more likely to contain bugs.

%% In addition we would like to add additional software projects to the
%% database. Another goal is to release the data to the community so others may
%% extend and add to it. We believe there are much more interesting results which
%% could be found using our database.

%% \section{Conclusion}
%% \label{sec-conclusion}

%% Resolving bugs represent a substantial amount of time and cost in software
%% projects. It is important to investigate the cause of bugs in order to reduce
%% the amount of bugs which occur. We believe that the data we have collected will
%% be beneficial for the software quality assurance and software developing
%% personnel to use the correlations we found, and investigate other possible
%% correlations from the data we collected and stored in a database. From manually
%% checking a random sample we found our data to be representative. Our data has
%% revealed that Thursdays have a high bug introduction rate relative to the total
%% number of commits for that day, while Mondays have the lowest bug introduction
%% rates. There may be many reasons behind this correlation, but we believe it
%% might be that people are being worn out towards the end of the week, and as such
%% are more likely to introduce bugs, except this would mean that Fridays would be
%% worst - which is not the case. It could also be that on Fridays people are more
%% alert and focused because they are excited for the weekend. It was strange that
%% Mondays yielded the best results for time to code, but this could be the case
%% because people are just back from the weekend and they are fully charged, ready
%% to work. Another possible theory for Monday's results, which is that people
%% delegate easier tasks for Mondays because its the start of the week, or spend
%% more time planning and laying out templates of what they will be coding, and as
%% such are not as prone to introducing bugs. This of course could be proven by
%% checking the size of the commits and what kind of changes are being made on
%% Mondays. Our data also showed that coding between 12AM and 9AM is a bad idea,
%% while the best time is between 11 AM and 3 PM. This is not a very surprising
%% result, considering how most users sleep during these hours, and if they happen
%% to be doing any coding during these hours, it would be considered to be outside
%% of their usual working hours - making them more prone to errors. From our
%% commits that were organized by author classification, it seems that daily
%% committers are less prone to bug introduction than those who are classified as
%% day job users. One theory that could potentially explain this result is that
%% hobbyist working on open source projects do it because they are highly
%% interested, while day job users are doing it to get payed, so the difference in
%% motivations could negatively affect the performance of the developer. Finally
%% our data shows that bug lifetimes decay exponentially, and on average are longer
%% for larger projects. This result of course is positive because it indicates that
%% the majority of bugs are dealt with fairly quickly.
